%%%%%%%%%%%%%%% INTRO %%%%%%%%%%%%%%%%%%%%
Slides 4,5,6,7,8

SLIDE 5

In the context of photonics and optics, we are interested in investigating the optical excitations of matter and their applications

Optics encompasses:
    Lasing,
    Polaritonics,
    Sensoring,
    others

Within solid matter and a classical description of light--matter int., photonics or classical electrodynamics
we typically divide materials into Metals or Dielectrics

Metals are characterized by
-Having free electronic charges
-Which makes them good conductors
-Capable of supporting collective electronic oscillations called plasmons: which underlie the entire field of plasmonics

Dielectrics
-have bound charges
-Bad at conducting electricity
-optical excitations consist of the bound charges oscillating around their equillibrium position, classicaly speaking
-In this dissertation we focus on the dipolar excitation called exciton: bound electron-hole states, can only be explained in the light of quantum principles,
confer strong optical resonances to insulators and semiconductors

SLIDE 6

Excitons are important for:

-Studying light--matter interactions in different regimes of coupling strength and develop new applications

-To experimentally build single-photon sources, which can be enhanced by putting the excitonic material inside a microcavity
to fabricate on-chip coherent light sources

-Induce nonlinear optical phenomena, such as the excitonic shift-current

-Useful in the field of valleytronics by enhancing the degree of valley polarization by stacking layers of excitonic materials, resulting in intensified circular dichroism, which can be used
for improving error detection in optical communications, or study chiral molecules of biological relevance, like proteins and nucleic acids

SLIDE 7

In 2D materials, excitons are a field by itself

-3D systems, excitons well described by the Rydberg series with modified mass and permittivity

-But for 2D materials, where epsilon is two-point function depending on r and r' separately, exciton series != Rydberg series,
improve the theory

-Account for non-local effects in the dielectric response of the material

-Back in 2014 for WS2 exciton series != Rydberg series, improved theory improves the agreement

-Extensive literature for excitons in 2D

SLIDE 7

-A typical example of gapped 2D material is hexagonal boron nitride

-Extensively used in photonics in its bulk form

-In bulk it's interesting in photonics for being a hyperbolic material in the Mid Infra-red

-However, as a monolayer, exhibits a sharp optical resonance in the UV, can only be explained by excitonic effects

SLIDE 8

-One motivation, besides the applications of excitons in 2D material, is a description of screening beyond the Rytova--Keldysh model, which is classical


Rytova-Keldysh
Advantages:
- Simple model, Taylor series up to leading order in q
- Analytical V(r)

Disadvantages
- Even for the simplest description of excitons, no analytical solution has been found due to the complicate form of the potential in real space
- Not appropriate for systems that are not atomically thin
- r0 has to be determined from first-principles after all

Ab-initio
Adv.:
-Any kind of system, single or many layers
-Plenty of packages to choose from
-Exact q-dependence of dielectric properties of any periodic system, one can even include dynamical effects

Disadvantages:
-Discarded in second-principles works for being too heavy
-Theorical basis is involded and not trivial (many-body perturbation theory)
-As far as we know, there is no direct 2D of implementation for screening,
2D systems are artificially made 3D to run the code

During my research stay abroad at the Universidad Autónoma de Madrid, I developed a strict 2D
implementation much more efficient than the typical 3D first-principles implementations, which
also permits to recover the Rytova-Keldysh limit and calculate the screening parameter.
Before discussing that, let us see what we can do with the RK model, valid in the low-q limit.

During the PhD studies we have found a middle ground to discuss later in this presentation


%%%%%%%%%%%%%%% hBN Superlattice %%%%%%%%%%%%%%%%%%%%
SLIDE 10

-In this part we discuss the results in this publication, where we investigate a hBN monolayer deposited on a quartz substrate, in contact with a patterned grating creating a 1D periodic electrostatic potential,
 that we model by a single-harmonic oscillating potential, on top of a Dirac Hamiltonian with mass to describe the low-energy properties of hBN

-Through an unitary transformation, we obtain an effective Hamiltonian with renormalized parameters, which induces an anisotropy in the dispersion, and a reduced bandgap

-Given in terms of the dimensionless parameter beta


SLIDE 11

-To determine the excitonic effects in this new material, we adopt a Wannier description of the electron-hole two-body problem
as we take the low-energy limit in the Hamiltonian, for which the band dispersion is quadratic

-The excitonic states and energies are obtained using the variational method with selected trial wavefunctions for the first four states

-We also examined the electronic bandgap, which is reduced as we increase beta

-The binding energies typically decrease with increasing beta, the anisotropy lifts the degeneracy between the x and y states

SLIDE 12

-In the excitonic optical conductivity, modeled by a sum of Lorentzians according to Thomas Pedersen, and consider two light-polarizations,
along the x direction along the periodicity of the potential, and the y direction along which the potential is invariant

-We examine the real and imaginary parts of the conductivity separately, in the top and bottom panels,
-On the left for x polarization, on the right for y polarization

-Where we can see that for increasing beta, in each panel from right to left, enhanced resonances for x pol. but diminished for y polar.
- beta=2.3 is qualitatively different for x pol. as Im{sigma} has only 1 zero, relevant later on.

SLIDE 13

-Conducting surface leads to radiation absorption by the material that I calculated using Fresnel equations, to observe that

-Can tune the absorption by tuning the potential;
-Induce a 2s peak higher than a 1s peak, unusual;
-To better understand the evolution in the absorption resonances, we plot for varying beta the absorption maxima
-Different transmission of light for different polarizations, working principle of a grid polarizer
-Non-trivial dependence of the absorption for different polarizations and excitonic states

SLIDE 14

-Finally, we study the emergent polaritonic effects of the nanostructred hBN layer
-The polariton, which is a surface confined EM mode with 2 different possible polarizations
-Focusing on the TM modes, propagation along the x directions;
-We obtained light modes for three different values of beta, with dispersion very distinct from that of free light with much higher confinement degrees
-Specially the most anisotropic case, with a dispersion back-bend due to the peak in the conductivity around the resonance due to the 2s state

%%%%%%%%%%%%%%% THE DIELECTRIC FUNCTION %%%%%%%%%%%%%%
In this part we will better understand the limits in the screening properties of insulators leading to the RK model using the Xatu code.

SLIDE 17
-We briefly revisit the microscopic theory for the dielectric function,
whose form for the potential depends on the dimensionality of the system

-For convenience, we used the symmetrized version of the diele. f. as explained in the thesis,
and the expression for the polarizability is also 2D


SLIDE 18

-The numerical dielectric function is used to model the screened potential;

-From which we can define a macroscopic 2D diele. f. as
-and define a macros. potential in the standard way

-We will see that in the low-q limit of the macr. diele. f. we recover the RK model
-3rd Linear response

SLIDE 19

-Unlike usually done in ab-initio implementations, where a plane-wave basis is used for the single-particle states
-Herein we work in a basis of localized orbitals, atomic or Wannier, where we use the LCAO to write the Bloch states

-Single-particle eigenstates of the Hamiltonian, which in a basis of localied orbitals, with the Tight-binding approximation,

-The eigenstates eigenenergy identity is translated to an eigenvector eigenvalue identity for the Bloch Hamiltonian,
representing the Hamiltonian of the crystal and the eigenvectors, the capital C's, store the TB coefficients

SLIDE 20

- We can quite efficiently compute the matrix elements of the polarizability by evaluating the plane-wave matrix elements
in the point-like orbital approximation, where no info about the atomic orbitals enters, so that the Fourier transform is
evaluated on the fly instantly

SLIDE 21

- To compute the excitons, as pre-processing steps

1)...
2)...
3)...
4) Finally compute the exciton

SLIDE 22
-Test example
-Monolayer hBN again, this time with a DFT model in a basis of localized Gaussian orbitals, obtained by DFT calculations using the CRYSTAL package

SLIDE 23
-We analise how does the inverse dielectric matrix behave as we increase its size
-We have in each panel a selected element of the inverse dielectric matrix, to see that it converges quite slowly

-Nonetheless, let us proceed with what we have.

SLIDE 24
-Trying to reproduce results in the literature
-We compare the 2D macroscopic dielectric function with our implementation and an ab initio one
to see good agreement, especially for lower values of q

Also, we get the RK model back, and an universal fact of e^2D(a)

SLIDE 25
In comparison with ab initio methods to summarize

SLIDE 26
To see the effect of the numerical screening in the calculations,
we incorporate the numerical inverse dielectric matrix in the calculation of the interaction terms
in the Bethe-Salpeter equation, describing the excitonic problem in momentum space
We discard the exchange term after all

SLIDE 27
-We have in the excitonic binding energy of the excitonic GS a competition between the cutoff for the dielectric matrix and the cutoff for the sum in D in the previous slide

-Fixing G^X_c = 0, the binding energy increases-> due to overscreening in the interaction
-Some columns E_X converges fixing the size of the dielectric matrix and including more terms in the sum, however not in general
-But in the diagonal elements we see a convergence, at least apparently

SLIDE 28
-Something similar for MoS2

SLIDE 29
-We can improve the dielectric response by adopting a quasi-2D approach

SLIDE 30
-We can define an eff. 2D macros. diele. f. by averaging along the monolayer thickness

-And show that the strict 2D diele. f. is the Q2D one in the 0 thickness limit

SLIDE 31
-To see that the Q2D result is in excellent agreement with the ab initio one

SLIDE 33