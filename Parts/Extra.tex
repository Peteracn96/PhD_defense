%!TEX root = ../main.tex %
% !TeX spellcheck = en_US

%%%%%%%  EXTRA  %%%%%%%%%

%\section{Supplementary Material}
%%%%%%%%%%%%%%%%%%%%%%%%%%%%%%%%%%%%%%%%%%%%%%%%%%%%%%%%%%%%%%%%%%%%

%%%%%%%%%%%%%%%%%%%%%%%%%%%%%%%%%%%%%%%%%%%%%%%%%%%

\begin{frame}{Screened potential matrix elements}

    \begin{equation}
        W_{\bG \bG'} (\bq) = \sqrt{v_c(\bq + \bG)}  \varepsilon^{-1}_{\bG \bG'} (\bq) \sqrt{v_c(\bq + \bG')}
    \end{equation}

    Regularizing the singular terms

    \begin{equation}
        W_{\bzero \bzero} (\bzero) = \frac{1}{\mathcal{A}_{\Gamma}} \int_{\mathcal{A}_{\Gamma}} \mathrm{d} \bq \, \varepsilon^{-1}_{\bzero \bzero} (\bq) v_c(\bq) \approx \frac{1}{\mathcal{A}_{\Gamma}} \int_{\mathcal{A}_{\Gamma}} \mathrm{d} \bq \, v_c(\bq) \big(1 - r_0 q \big)
    \end{equation}

    and for the wing terms

    \begin{equation}
        W_{\bG \bzero} (\bzero) = W_{\bzero \bG} (\bzero) = 0
    \end{equation}
\end{frame}


\begin{frame}{Q2D dielectric function}

   \begin{figure}[h]
            \centering
            \includegraphics[width=0.85\textwidth]{Figures/epsilon_Q2D_MoS2_unjoined.png}        
        \end{figure}
\end{frame}

\begin{frame}{Computational Framework:\\ Pre-processing calculations}

    \begin{itemize}
        \item DFT w/ Quantum ESPRESSO + Wannier90 (\texttt{\_tb.dat} file)
        \\ OR 
        \item DFT w/ CRYSTAL (\texttt{.oupt} file)
    \end{itemize}
    
    \,\,\,\,\,\,\,\,\,\,\,\,\,\,\Big\downarrow

    System file:
    \begin{itemize}
        \item Wannier90 \texttt{<filename>\_tb.dat} file $\xrightarrow{\text{w90 utility}}$ \texttt{.model} file as input for xatu
        \item CRYSTAL \texttt{<filename>.oupt} file as input for xatu
    \end{itemize}
\end{frame}

\begin{frame}{Monolayer \ch{MoS2} Band Structure}

    \begin{figure}[H]
        \centering
        \includegraphics[width=0.6\linewidth]{Figures/MoS2_bands}
        \caption{Band structure of monolayer \ch{MoS2} using CRYSTAL}
    \end{figure}

\end{frame}


\begin{frame}{Excitons in \ch{MoS2}: Numerical Results}

    \begin{table}[h]
        \caption{This table examines the convergence of the excitonic ground state binding energy with the cutoff for the dielectric matrix, $G_c^{\varepsilon}$, and for the interaction matrix elements, $G_c^X$, always with $G_c^X < G_c^{\varepsilon}$. We have used $N_k=60^2$, $N_c=N_v=1$, and we have excluded the exchange interaction term. For the size of the regularization region, we used the radius $q_0 = 0.6 k_0$, where $k_0$ is the norm of the wavevector(s) closest to the origin. All values are in eV. $\Delta = 2.08366$~eV}
        \centering
            \begin{tabular}{cc||cccccc}
                \cline{3-8}
                &   & \multicolumn{6}{c|}{$G^{\varepsilon}_c$ (\AA$^{-1}$)}                                            \\ \cline{3-8}
                &   & 0                      & 3                      & 4                           & 5                        & 7                      & 8                       \\ \cline{8-8} \hline \hline
                \multicolumn{1}{|c|}{\multirow{4}{*}{$G^X_{c}$}} & 0 & \multicolumn{1}{c|}{X} & 0.979507               & 1.567935                    & 1.401440                 & 1.951914               & ---        \\
                \multicolumn{1}{|c|}{}                           & 3 & \multicolumn{1}{c|}{X} & 0.756373               & 1.619208                    & 1.785953                 & ---                    & ---          \\ \cline{4-4} \cline{4-4}
                \multicolumn{1}{|c|}{}                           & 4 & X                      & \multicolumn{1}{c|}{X} & 0.774599                    & 1.105537                 & ---                    & ---          \\ \cline{5-5} \cline{5-5}
                \multicolumn{1}{|c|}{}                           & 5 & X                      & X                      & \multicolumn{1}{c|}{X}      & 0.778244                 & ---                    & ---          \\ \cline{6-6} \cline{6-6}
                \multicolumn{1}{|c|}{}                           & 7 & X                      & X                      & X                           & \multicolumn{1}{c|}{X}   & 0.792309               & 1.300835          \\ \cline{7-7}
                \multicolumn{1}{|c|}{}                           & 8 & X                      & X                      & X                           & X                        & \multicolumn{1}{c|}{X} & 0.799489 \\  \cline{1-2} \cline{1-2} \cline{8-8}
            \end{tabular}
    \end{table}

    Cell with -- means that the output result does not make physical sense

    At least apparent numerical convergence
\end{frame}