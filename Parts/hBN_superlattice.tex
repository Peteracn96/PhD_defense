%!TEX root = ../main.tex %
% !TeX spellcheck = en_US

%%%%%%%  hBN Superlattice  %%%%%%%%%

\section{Part I: Exciton--Polaritons in a 1D hBN Superlattice}
%%%%%%%%%%%%%%%%%%%%%%%%%%%%%%%%%%%%%%%%%%%%%%%%%%%%%%%%%%%%%%%%%%%%

\subsection{Setup}

%%%%%%%%%%%%%%%%%%%%%%%%%%%%%%%%%%%%%%%%%%%%%%%%%%%%%%%%%%%%%%%%%%%

\begin{frame}[t]
\frametitle{hBN under an external periodic potential}


\begin{columns}[T]

        \column{.6\linewidth}
        
            \begin{textblock*}{10.0cm}(0.2cm,0.5cm)
            \only<1->{\begin{figure}
                \includegraphics[scale=0.8]{Figures/Work_on_hBN_perspective_2.pdf}
            \end{figure}
            \small
            \underline{P.~Ninhos} et. al - \textit{ACS Nano}  \textbf{18} 31 (2024)}
            \vspace{0.3cm}
            \only<4->{For $\beta ~1$, and $V_0~$...\textcolor{red}{Ver na reply letter}}
            \end{textblock*}
        

        \column{.4\linewidth}

        \only<2->{
            \begin{textblock*}{4.0cm}(9.0cm,0.6cm)
            \small
            \begin{equation*}
                \begin{aligned}
                & H_0 = H_{\mathrm{Dirac}} + V_0 \cos (G_0 x), \, \vectormath{G}_0 = \frac{2 \pi}{L} \mathbf{\hat{x}} \\
                 \Rightarrow & \overline{H}_0 = \hbar v_F [q_{x} \sigma_{x} +J_{0}(\beta)q_{y} \sigma_{y}] - \frac{E_g}{2} J_{0}(\beta) \sigma_{z}
                \end{aligned}
            \end{equation*}
            \small \vspace{-0.2cm}
            \begin{equation*}
                \beta = V_{0} L/(\pi\hbar v_{F})
            \end{equation*}
            \end{textblock*}
        }

        \only<3->{
            \vspace{1.6cm}
            \begin{figure}
                \centering
                \includegraphics[scale=0.6]{Figures/anisotropic_bands.pdf}
            \end{figure}
        }
    \end{columns}
    


\end{frame}

\subsection{Excitonic States}

%%%%%%%%%%%%%%%%%%%%%%%%%%%%%%%%%%%%%%%%%%%%%%%%%%%%%%%%%%%%%%%%%%%%%%%%%%%%%%%%%%%%%%%%%%%%%%%%

\begin{frame}{Wannier equation and variational method}


\only<1->{

\begin{textblock*}{6.0cm}(3.0cm,1.3cm)
\centering{Wannier equation} [1]
\end{textblock*}

\begin{textblock*}{10.0cm}(0.8cm,1.3cm)
\centering{
\begin{empheq}[innerbox=\fbox]{align*}
E_{bind.,\nu} \psi_\nu (\vectormath{r}) = \left(\frac{p_{x}^{2}}{2\mu_{x}}+\frac{p_{y}^{2}}{2\mu_{y}}\right) \psi_\nu (\vectormath{r}) - V_{\mathrm{RK}}(\vectormath{r}) \psi_\nu (\vectormath{r}) 
\end{empheq}
}
\end{textblock*}

\begin{textblock*}{12.0cm}(0.2cm,8.0cm)
\scriptsize{[1] M. Vasilevskiy et al. - \textit{J. Phys.: Condens. Matter} \textbf{34} 045702 (2022)}
\end{textblock*}

}

\only<2->{
    \begin{textblock*}{5.0cm}(11.cm,1.cm)
    \small{
        \begin{equation*}
            V_{\mathrm{RK}}(q)=\frac{e^{2}}{2 \varepsilon_0 (\kappa+q r_{0}) q}
        \end{equation*}
        
        \begin{equation*}
            \kappa = \frac{\varepsilon_1 + \varepsilon_2}{2}
        \end{equation*}}
    \end{textblock*}
}


\only<3->{

\begin{textblock*}{5.0cm}(.4cm,3.8cm)
\underline{Trial wavefunctions}:
\end{textblock*}

\begin{textblock*}{5.0cm}(.cm,3.9cm)
\centering{
\begin{equation*}
\begin{split}
   \Psi_{1s}&=C_{1s}\ee^{-\rho_{1s}} \\
   \Psi_{2x}&=C_{2x}x\ee^{-\rho_{2x}} \\
   \Psi_{2y}&=C_{2y}y\ee^{-\rho_{2y}} \\
   \Psi_{2s}&=C_{2s}\left(1-d\rho_{2s}\right)\ee^{-\rho_{2s}} \\
   \rho_{\nu}=&\sqrt{a_{\nu}x^{2}+b_{\nu}y^{2}}\\
   \nu&=1s,2x,2y,2s
\end{split}
\end{equation*}}
\end{textblock*}
}

\only<4->{
    \begin{textblock*}{10.0cm}(4.8cm,3.cm)
        \centering{
        \begin{center}
        \includegraphics[scale=0.45]{Figures/Band_gap_and_binding_energies.pdf}
        \end{center}}
    \end{textblock*}
}

\end{frame}

%%%%%%%%%%%%%%%%%%%%%%%%%%%%%%%%%%%%%%%%%%%%%%%%%%%%%%%%%%%%%%%%%%%%%%%%%%%%%%

\subsection{Optical Response}

%%%%%%%%%%%%%%%%%%%%%%%%%%%%%%%%%%%%%%%%%%%%%%%%%%%%%%%

\begin{frame}{Optical Conductivity}

% \only<1->{
% \begin{textblock*}{5.0cm}(.2cm,0.8cm)
% \centering{\small
% \begin{equation*}
%     \sigma (\omega) = \frac{\sigma_0}{i} \sum_\alpha E_\alpha \frac{|\Omega_\alpha|^2 /32 \pi^3}{E_\alpha - (\hbar \omega+i\Gamma)}
% \end{equation*}
% \vspace{-0.3cm}
% \begin{equation*}
%   \Omega_\alpha =  \sum\nolimits_{\mathbf{k}} \Psi_{\alpha} (\mathbf{k}) \langle v, \mathbf{k} | \vectormath{r} \cdot \hat{\mathbf{e}} | c, \mathbf{k} \rangle
% \end{equation*}}
% \end{textblock*}
% }

\only<1->{
\begin{textblock*}{6.0cm}(10.3cm,1.0cm)
\centering{\small
According to Pedersen [1]
\begin{equation*}
    \scalebox{1.0}{$\sigma_{jj} (\omega) = -i \sigma_0 \sum\limits_{\nu=1s,2s}  \frac{E_\nu |\Omega^j_\nu|^2 }{E_\nu - (\hbar \omega + i\Gamma)} $}
\end{equation*}}
\end{textblock*}
\begin{textblock*}{10.0cm}(.2cm,8.cm)
\scriptsize{[1] T.~G.~Pedersen, \textit{Phys. Rev. B}  \textbf{92} 235432 (2015)}
\end{textblock*}
}


\only<2->{
\begin{textblock*}{5.0cm}(10.5cm,2.2cm)
\centering{
\begin{equation*}
    \begin{aligned}
        &\scalebox{.8}{$\Omega_\nu =  \sum_{\bm{q}} \Psi_{\nu} (\bm{q}) \langle -, \bm{q} | \vectormath{r} \cdot \hat{\mathbf{e}}_j | +, \bm{q} \rangle$} \\
        &\scalebox{.8}{$\mathbf{\hat{e}}_j = \mathbf{\hat{x}}, \mathbf{\hat{y}}$} \\
        &\scalebox{.8}{$\overline{H}_0 |\pm, \bq \rangle =  E_{\pm,\bq}|\pm, \bq \rangle$}
    \end{aligned}
\end{equation*}}
\end{textblock*}
}

\only<3>{
\begin{textblock*}{10.0cm}(0.1cm,0.8cm)
\begin{figure}
    \centering
    \includegraphics[scale=0.4]{Figures/sigma_grid_slides.pdf}
\end{figure}
\end{textblock*}
}

\only<4>{
\begin{textblock*}{10.0cm}(0.1cm,0.8cm)
\begin{figure}
    \centering
    \includegraphics[scale=0.4]{Figures/sigma_grid_Re_parts.pdf}
\end{figure}
\end{textblock*}
}

\only<5>{
\begin{textblock*}{10.0cm}(0.1cm,0.8cm)
\begin{figure}
    \centering
    \includegraphics[scale=0.4]{Figures/sigma_grid_Im_parts.pdf}
\end{figure}
\end{textblock*}
}

\only<6-8>{
\begin{textblock*}{10.0cm}(0.1cm,0.8cm)
\begin{figure}
    \centering
    \includegraphics[scale=0.4]{Figures/sigma_grid_x_pol.pdf}
\end{figure}
\end{textblock*}
}

\only<9>{
\begin{textblock*}{10.0cm}(0.1cm,0.8cm)
\begin{figure}
    \centering
    \includegraphics[scale=0.4]{Figures/sigma_grid_y_pol.pdf}
\end{figure}
\end{textblock*}
}


\begin{textblock*}{10.0cm}(0.1cm,0.8cm)
\only<10>{\begin{figure}
    \centering
    \includegraphics[scale=0.4]{Figures/sigma_grid_slides.pdf}
\end{figure}}
\end{textblock*}

\begin{textblock*}{8.0cm}(10.cm,4.7cm)
    \begin{itemize}
        \item<7-> Increasing peak intensity for $xx$
        \item<8-> $\Im\{\sigma\}$ w/ $\beta = 2.3$ only one zero
        \item<9-> $\hat{\mathbf{e}}_j = \mathbf{\hat{y}}$, $\sigma_{yy} (\omega)$ is attenuated
    \end{itemize}
\end{textblock*}


\end{frame}


%%%%%%%%%%%%%%%%%%%%%%%%%%%%%%%%%%%%%%%%%%%%%%%%%%%%%%%%%%%%%%%%%%%%%%%%%%%%%
\begin{frame}{Optical absorption}

\only<1->{
\begin{textblock*}{5.0cm}(.2cm,0.8cm)
\small{
\begin{equation*}
    \mathcal{A}_{\mathrm{opt}} = 1 - \mathcal{R} - \frac{\Re\{\sqrt{\varepsilon_2}\}}{\Re\{\sqrt{\varepsilon_1}\}} \mathcal{T}\,, \quad \mathcal{R} = \left| \frac{\sqrt{\varepsilon_2} - \sqrt{\varepsilon_1} + \frac{\sigma}{\varepsilon_0 c}}{\sqrt{\varepsilon_2} + \sqrt{\varepsilon_1} + \frac{\sigma}{\varepsilon_0 c}} \right|^2\,, \quad \mathcal{T} = \left| \frac{2\sqrt{\varepsilon_1}}{\sqrt{\varepsilon_2} + \sqrt{\varepsilon_1} + \frac{\sigma}{\varepsilon_0 c}} \right|^2
\end{equation*}}
\end{textblock*}
}


\begin{textblock*}{12.0cm}(2.0cm,2.2cm)
    \only<2>{
    \begin{figure}
        \centering
        \begin{subfigure}{.5\textwidth}
            \centering
            \includegraphics[scale=0.45]{Figures/absorption_of_omega_a.pdf}
        \end{subfigure}%
        \begin{subfigure}{.5\textwidth}
            \centering
            \transparent{0.5}\includegraphics[scale=0.45]{Figures/absorption_of_omega_b.pdf}
        \end{subfigure}
    \end{figure}}
    \only<3>{\begin{figure}
        \centering
        \begin{subfigure}{.5\textwidth}
            \centering
            \includegraphics[scale=0.45]{Figures/absorption_of_omega_a.pdf}
        \end{subfigure}%
        \begin{subfigure}{.5\textwidth}
            \centering
            \transparent{0.5}\includegraphics[scale=0.45]{Figures/absorption_of_omega_b.pdf}
        \end{subfigure}
    \end{figure}}
    \only<4>{
    \begin{figure}
        \centering
        \begin{subfigure}{.5\textwidth}
            \centering
            \transparent{0.5}\includegraphics[scale=0.45]{Figures/absorption_of_omega_a.pdf}
        \end{subfigure}%
        \begin{subfigure}{.5\textwidth}
            \centering
            \includegraphics[scale=0.45]{Figures/absorption_of_omega_b.pdf}
        \end{subfigure}
    \end{figure}}
    \only<5>{
        \begin{figure}
            \centering
            \begin{subfigure}{.5\textwidth}
                \centering
                \includegraphics[scale=0.45]{Figures/absorption_of_omega_a.pdf}
            \end{subfigure}%
            \begin{subfigure}{.5\textwidth}
                \centering
                \includegraphics[scale=0.45]{Figures/absorption_of_omega_b.pdf}
            \end{subfigure}
        \end{figure}}
\end{textblock*}


\begin{textblock*}{7.0cm}(8.0cm,6.9cm)
\only<4->{
    \begin{itemize}   
        \item For $y$ pol.~absorption is suppressed
    \end{itemize}}
\end{textblock*}

\begin{textblock*}{8.0cm}(0.2cm,6.6cm)
\only<3->{
    \begin{itemize}
        \item Tunable absorption 
        \item Non-monotonic behaviour of the absorption
        \item $2s$ peak higher than $1s$ peak
    \end{itemize}}
\end{textblock*}

\end{frame}


\begin{frame}{Absorption peaks}

    \only<1->{
\begin{textblock*}{6.0cm}(.2cm,0.9cm)
\small{
    \begin{empheq}[innerbox=\fbox,right={\,, \omega = E_\nu/\hbar\,, \nu=1s,2s}]{align}
        \max_{\omega}{\mathcal{A}_{\mathrm{opt}}} = \frac{4 \sqrt{\varepsilon_1} \sigma'}{(\sqrt{\varepsilon_2} + \sqrt{\varepsilon_1} + \sigma')^2}\ \notag
    \end{empheq}}
\end{textblock*}
}

\only<2->{
\begin{textblock*}{6.0cm}(0.8cm,2.3cm)
\begin{figure}
    \centering
    \includegraphics[scale=0.5]{Figures/absorption_of_beta_a.pdf}
\end{figure}
\end{textblock*}
}

\only<3->{
\begin{textblock*}{6.0cm}(7.2cm,2.3cm)
\begin{figure} % change to subfigure environment
    \centering
    \includegraphics[scale=0.5]{Figures/Absorption_vs_Resigma.pdf}
\end{figure}
\end{textblock*}
}

\only<4->{
\begin{textblock*}{14.0cm}(0.8cm,6.7cm)
\begin{itemize}
    \item Tunable absorption 
    \item Non-monotonic behaviour of the absorption peak
    \item Different response for different polarizations $\Rightarrow$ grid polarizer
\end{itemize}
\end{textblock*}
}

    
\end{frame}


%%%%%%%%%%%%%%%%%%%%%%%%%%%%%%%%%%%%%%%%%%%%%%%%%%%%%%%%%%%%%%%%%%%%%%%%%%%%%%%

\subsection{Exciton--Polaritons}
%%%%%%%%%%%%%%%%%%%%%%%%%%%%%%%%%%%%%%%%%%%%%%%%%%%%%%%

\begin{frame}{Exciton--polaritons}
\setbeamercovered{transparent=15}
\only<1->{

\begin{textblock*}{6.0cm}(0.cm,1.2cm)
\begin{center}
\vspace{-0.5cm}
\hspace{1.5cm}
\begin{tikzpicture}[baseline={(0, -2.0)},scale=1.5,>=Stealth]
\def\r{0.2cm}
\definecolor{darkgreen}{RGB}{65,117,5}
\begin{scope}[xshift=-7.*\r,yshift=0]

\draw[black] (2.5*\r,4.5*\r) node {Transverse Electric (TE)};

\draw[black] (2.5*\r,-3.1*\r) node {\small$\text{Im}\{\sigma(\omega)\}<0$};

%wavevector
\draw[->,color=black] (\r,0) -- ({5*\r},0);
\draw[black] (2.5*\r,-0.7*\r) node {$\vec{q}$};

%Electric field
\draw[color=red!80!black](0,0) circle (\r);
\draw[color=red!80!black](-{\r*sqrt(2)/2},-{\r*sqrt(2)/2}) -- ({\r*sqrt(2)/2},{\r*sqrt(2)/2});
\draw[color=red!80!black](-{\r*sqrt(2)/2},{\r*sqrt(2)/2}) -- ({\r*sqrt(2)/2},-{\r*sqrt(2)/2});
\draw[red!80!black] (0.,-1.7*\r) node {$\vec{E}$};

%Magnetic field
\draw[->,color=darkgreen] ({\r*cos(20)},{\r*sin(20)}) -- ({6*\r*cos(20)},{6*\r*sin(20)});
\draw[darkgreen] ({-0.2*\r+6*\r*cos(20)},{0.8*\r+6*\r*sin(20)}) node {$\vec{H}$};
\end{scope}

\begin{scope}[shift={(1.3cm,0cm)}]

\draw[black] (2.5*\r,4.5*\r) node {Transverse Magnetic (TM)};
\draw[black] (2.5*\r,-3.1*\r) node {\small$\text{Im}\{\sigma(\omega)\}>0$};

%wavevector
\draw[->,color=black] (\r,0) -- ({5*\r},0);
\draw[black] (2.5*\r,-0.7*\r) node {$\vec{q}$};


%Magnetic field
\draw[color=darkgreen](0,0) circle (\r);
\draw[color=darkgreen](-{\r*sqrt(2)/2},-{\r*sqrt(2)/2}) -- ({\r*sqrt(2)/2},{\r*sqrt(2)/2});
\draw[color=darkgreen](-{\r*sqrt(2)/2},{\r*sqrt(2)/2}) -- ({\r*sqrt(2)/2},-{\r*sqrt(2)/2});
\draw[darkgreen] (0.,-1.7*\r) node {$\vec{H}$};

%Electric field
\draw[->,color=red!80!black] ({\r*cos(20)},{\r*sin(20)}) -- ({6*\r*cos(20)},{6*\r*sin(20)});
\draw[red!80!black] ({-0.2*\r+6*\r*cos(20)},{0.8*\r+6*\r*sin(20)}) node {$\vec{E}$};
\end{scope}
 


%\draw[->,blue] (0,0) -- (2/3,{2/sqrt(3)});
%\draw[->,blue] (0,0) -- (2/3,-{2/sqrt(3)});

%\draw (2/3,{2/sqrt(3) - 0.3}) node {$\color{blue} \boldsymbol{b}_1$};
%\draw (2/3,{-2/sqrt(3) + 0.3}) node {$\color{blue} \boldsymbol{b}_2$};

\end{tikzpicture}
\end{center}

\end{textblock*}


\begin{textblock*}{8.0cm}(7.7cm,.2cm)
\begin{center}
    \includegraphics[scale=0.2]{Figures/polariton_propagation.png}
\end{center}

\end{textblock*}

}

\only<2->{
\begin{textblock*}{6.0cm}(-0.5cm,3.9cm)

\begin{equation*}
    \scalebox{.8}{$\kappa_1 + \kappa_2 - i \omega \mu_0 \sigma_{yy}(\omega) = 0$}
\end{equation*}

\end{textblock*}

\begin{textblock*}{6.0cm}(3.0cm,3.9cm)

\begin{empheq}[innerbox=\fbox]{align}
    \scalebox{.8}{$\frac{\varepsilon_1}{\kappa_1} + \frac{\varepsilon_2}{\kappa_2} + i  \frac{\sigma_{xx} (\omega)}{\varepsilon_0 \omega} = 0$} \notag
\end{empheq}

\end{textblock*}

\begin{textblock*}{8.5cm}(5.8cm,3.9cm)

\begin{equation*}
    \scalebox{.8}{$\kappa_j = \sqrt{q^2 - \varepsilon_j(\omega) \frac{\omega^2}{c^2}},\, j=1,2$}
\end{equation*}
%stress that we're looking only at TM modes
%stress different degrees of confinement micrometer vs nanometer
\end{textblock*}
}

\begin{textblock*}{12cm}(1.6cm,4.5cm)
\setbeamercovered{transparent=15}

\begin{columns}[b]
\begin{column}{0.33\textwidth}
 \only<2,3,6>{
   \begin{figure}
    \centering
        \includegraphics[width=0.9\linewidth]{Figures/polariton_a.png}
   \end{figure}
  }      
 \end{column}
\begin{column}{0.33\textwidth}
 \only<2,4,6>{
    \hspace{0.5cm}
  \begin{figure}
    \centering
        \includegraphics[width=0.9\linewidth]{Figures/polariton_b.png}
   \end{figure}
   }
 \end{column}
 \begin{column}{0.33\textwidth}
 \only<2,5,6>{
  \begin{figure}
    \centering
        \includegraphics[width=0.9\linewidth]{Figures/polariton_c.png}
   \end{figure}
   }
 \end{column}
\end{columns}
\end{textblock*}
% \only<3>{
% \begin{textblock*}{5.0cm}(0.7cm,4.4cm)
% \begin{center}
%     \includegraphics[scale=0.25]{Figures/polariton_a.png}
% \end{center}
% \end{textblock*}
% \begin{textblock*}{5.0cm}(5.7cm,4.4cm)
% \begin{center}
%     \includegraphics[scale=0.25]{Figures/polariton_a.png}
% \end{center}
% \end{textblock*}
% \begin{textblock*}{5.0cm}(10.7cm,4.4cm)
% \begin{center}
%     \includegraphics[scale=0.25]{Figures/polariton_a.png}
% \end{center}
% \end{textblock*}
% }
    
% \only<3>{
% \begin{textblock*}{5.0cm}(0.7cm,4.4cm)
% \begin{center}
%     \includegraphics[scale=0.25]{Figures/polariton_a.png}
% \end{center}
% \end{textblock*}
% \begin{textblock*}{5.0cm}(5.7cm,4.4cm)
% \begin{center}
%     \includegraphics[scale=0.25]{Figures/polariton_a.png}
% \end{center}
% \end{textblock*}
% \begin{textblock*}{5.0cm}(10.7cm,4.4cm)
% \begin{center}
%     \includegraphics[scale=0.25]{Figures/polariton_a.png}
% \end{center}
% \end{textblock*}
% }
\end{frame}

\begin{frame}{Summary}

    \begin{textblock*}{8cm}(0.5cm,1.2cm)
        \begin{itemize}
            \item<1-> Tune the frequency of the optical resonances
        \end{itemize}
    \end{textblock*}

    \begin{textblock*}{8cm}(0.5cm,4.5cm)
        \begin{itemize}
            \item<2-> Tunable absorption by changing the polarization
        \end{itemize}
    \end{textblock*}

    \begin{textblock*}{8cm}(8.0cm,1.0cm)
        \only<1->{
        \begin{figure}
            \centering
            \includegraphics[scale=0.25]{Figures/absorption_of_omega_a.pdf}
        \end{figure}}
    \end{textblock*}
    \begin{textblock*}{8cm}(8.0cm,4.4cm)
        \only<2->{
        \begin{figure}
            \centering
            \includegraphics[scale=0.22]{Figures/absorption_of_beta_a.pdf}
        \end{figure}}
    \end{textblock*}

    \begin{textblock*}{8cm}(0.5cm,5.3cm)
        \only<3->{
        \begin{block}{}    
            However....
            \begin{itemize}\setlength{\itemsep}{6pt}
                \item<4-> Where does $\varepsilon_{\mathrm{RK}}(q)$ come from?
                \item<5-> How to determine $r_0$?
                \item<6-> How does $\varepsilon_{\mathrm{2D}}(q)$ look like for higher $q$?
            \end{itemize}
        \end{block}}        
    \end{textblock*}
\end{frame}
