%!TEX root = ../main.tex %
% !TeX spellcheck = en_US

%%%%%%%  CONCLUSIONS  %%%%%%%%%

\section{Conclusions}
%%%%%%%%%%%%%%%%%%%%%%%%%%%%%%%%%%%%%%%%%%%%%%%%%%%%%%%%%%%%%%%%%%%%

%%%%%%%%%%%%%%%%%%%%%%%%%%%%%%%%%%%%%%%%%%%%%%%%%%%%%%%%%%%%%%%%%%%%%%%%%%%%%%%

\begin{frame}{Conclusions: Part I}    
    \only<1>{
    \begin{textblock*}{10cm}(0.5cm,1.2cm)
        \begin{itemize}                
            \item hBN as a promising material for mid IR--UV polaritonics
        \end{itemize}
        
        \begin{itemize}                
            \item Tune the frequency of the optical resonances
        \end{itemize}
    \end{textblock*}

    \begin{textblock*}{10cm}(0.5cm,5.2cm)
        \begin{itemize}
            \item Linear dichroism
        \end{itemize}
    \end{textblock*}

    \begin{textblock*}{8cm}(.5cm,2.2cm)       
        \begin{figure}
            \centering
            \transparent{0.5}\includegraphics[scale=0.3]{Figures/absorption_of_omega_a.pdf}
        \end{figure}
    \end{textblock*}

    \begin{textblock*}{8cm}(.5cm,5.5cm)        
        \begin{figure}
            \centering
            \transparent{0.5}\includegraphics[scale=0.3]{Figures/absorption_of_beta_a.pdf}
        \end{figure}
    \end{textblock*}}


    \only<2>{
    \begin{textblock*}{10cm}(0.5cm,1.2cm)
        \begin{itemize}                
            \item \underline{hBN as a promising material for mid IR--UV polaritonics}
        \end{itemize}
        
        \begin{itemize}                
            \item Tune the frequency of the optical resonances
        \end{itemize}
    \end{textblock*}

    \begin{textblock*}{10cm}(0.5cm,5.2cm)
        \begin{itemize}
            \item Linear dichroism
        \end{itemize}
    \end{textblock*}

    \begin{textblock*}{8cm}(.5cm,2.2cm)       
        \begin{figure}
            \centering
            \transparent{0.5}\includegraphics[scale=0.3]{Figures/absorption_of_omega_a.pdf}
        \end{figure}
    \end{textblock*}

    \begin{textblock*}{8cm}(.5cm,5.5cm)        
        \begin{figure}
            \centering
            \transparent{0.5}\includegraphics[scale=0.3]{Figures/absorption_of_beta_a.pdf}
        \end{figure}
    \end{textblock*}}

    \only<3>{
    \begin{textblock*}{10cm}(0.5cm,1.2cm)
        \begin{itemize}                
            \item hBN as a promising material for mid IR--UV polaritonics
        \end{itemize}
        
        \begin{itemize}                
            \item \underline{Tune the frequency of the optical resonances}
        \end{itemize}
    \end{textblock*}

    \begin{textblock*}{10cm}(0.5cm,5.2cm)
        \begin{itemize}
            \item Linear dichroism
        \end{itemize}
    \end{textblock*}

    \begin{textblock*}{8cm}(.5cm,2.2cm)       
        \begin{figure}
            \centering
            \includegraphics[scale=0.3]{Figures/absorption_of_omega_a.pdf}
        \end{figure}
    \end{textblock*}

    \begin{textblock*}{8cm}(.5cm,5.5cm)        
        \begin{figure}
            \centering
            \transparent{0.5}\includegraphics[scale=0.3]{Figures/absorption_of_beta_a.pdf}
        \end{figure}
    \end{textblock*}}

    \only<4>{
    \begin{textblock*}{10cm}(0.5cm,1.2cm)
        \begin{itemize}                
            \item hBN as a promising material for mid IR--UV polaritonics
        \end{itemize}
        
        \begin{itemize}                
            \item Tune the frequency of the optical resonances
        \end{itemize}
    \end{textblock*}

    \begin{textblock*}{10cm}(0.5cm,5.2cm)
        \begin{itemize}
            \item \underline{Linear dichroism}
        \end{itemize}
    \end{textblock*}

    \begin{textblock*}{8cm}(.5cm,2.2cm)       
        \begin{figure}
            \centering
            \transparent{0.5}\includegraphics[scale=0.3]{Figures/absorption_of_omega_a.pdf}
        \end{figure}
    \end{textblock*}

    \begin{textblock*}{8cm}(.5cm,5.5cm)        
        \begin{figure}
            \centering
            \includegraphics[scale=0.3]{Figures/absorption_of_beta_a.pdf}
        \end{figure}
    \end{textblock*}}

    \onslide<5->{
    \begin{textblock*}{10cm}(0.5cm,1.2cm)
        \begin{itemize}                
            \item hBN as a promising material for mid IR--UV polaritonics
        \end{itemize}
        
        \begin{itemize}                
            \item Tune the frequency of the optical resonances
        \end{itemize}
    \end{textblock*}

    \begin{textblock*}{10cm}(0.5cm,5.2cm)
        \begin{itemize}
            \item Linear dichroism
        \end{itemize}
    \end{textblock*}

    \begin{textblock*}{8cm}(.5cm,2.2cm)       
        \begin{figure}
            \centering
            \includegraphics[scale=0.3]{Figures/absorption_of_omega_a.pdf}
        \end{figure}
    \end{textblock*}

    \begin{textblock*}{8cm}(.5cm,5.5cm)        
        \begin{figure}
            \centering
            \includegraphics[scale=0.3]{Figures/absorption_of_beta_a.pdf}
        \end{figure}
    \end{textblock*}}
    \setbeamercovered{transparent=15}
    \begin{textblock*}{6.2cm}(9.5cm,1.6cm)        
        
        \begin{exampleblock}{}
            \small    
            \onslide<5->{Of experimental relevance (recalling)}
            \setbeamertemplate{items}[triangle]
            \begin{itemize}\setlength{\itemsep}{6pt}                
                \item<6-> D.~R.~Danielsen et al. - \textit{ACS Nano} \textbf{19} 22 (2025) $\leftarrow$ experiment with \ch{WS2} % "Fourier-Tailored Light–Matter Coupling in van der Waals Heterostructures"
                \item<7-> G.~Ermolaev et al. - \textit{Arxiv} \textbf{2509} 18866 (2024) $\leftarrow$ experiment with \ch{CrSBr} % "Giant optical anisotropy in CrSBr from giant exciton oscillator strength"
                \item<8-> Possible applications: optical components, filters, polarizers
            \end{itemize}
        \end{exampleblock}
    \end{textblock*}
\end{frame}


\begin{frame}{Conclusions: Part II}

    \begin{columns}

        \column{0.6\textwidth}
            \begin{enumerate}\setlength{\itemsep}{10pt}  
            \item<1-> Point-like orbital approximation \emoji{thumbs-up}
            \item<2-> $\varepsilon^{\mathrm{2D}}(q)$ by direct inversion of $\varepsilon_{\bG \bG'} (\bq)$ \emoji{backhand-index-pointing-left}   
            \item<3-> Rytova--Keldysh $\leftarrow$ low-$q$ limit of $\varepsilon^{\mathrm{2D}}(q)$ \emoji{check-mark}
            \item<4-> exciton binding energy seems to converge \emoji{flexed-biceps}
            \item<5-> Panel \textbf{b} with ~95 points: \\
            $\sim$ 20 mins for 2D, $\sim$ 30 mins for Q2D \emoji{timer-clock}
        \end{enumerate}
        
               
        \column{0.43\textwidth}

        \onslide<2->{
        \begin{figure}[h]
            \centering
            \includegraphics[width=0.85\textwidth]{Figures/epsilon_M_vs_q_hBN_&_MoS2.pdf}        
        \end{figure}}

        
    \end{columns}

\end{frame}