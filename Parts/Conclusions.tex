%!TEX root = ../main.tex %
% !TeX spellcheck = en_US

%%%%%%%  CONCLUSIONS  %%%%%%%%%

\section{Conclusions}
%%%%%%%%%%%%%%%%%%%%%%%%%%%%%%%%%%%%%%%%%%%%%%%%%%%%%%%%%%%%%%%%%%%%

%%%%%%%%%%%%%%%%%%%%%%%%%%%%%%%%%%%%%%%%%%%%%%%%%%%%%%%%%%%%%%%%%%%%%%%%%%%%%%%

\begin{frame}{Summary}

\begin{itemize}

\setlength{\itemsep}{10pt}
    \item hBN as a promising material for UV polaritonics 
    \item 1D superlattice with tunable periodicity: from deep-UV to the visible
    \item Tunable absorption
    \item Different response for different polarizations
    \item Confined EM modes with tunable frequency
\end{itemize}


\begin{itemize}
    \item the point-like (and orthogonal) orbital approximation does not set us back
    \item $v_c(q) \sim 1/q$ brings numerical and practical benefits; Also, no need to truncate $v_c$
    \item 2D, Q2D and \textit{ab-initio} approaches agree in the very low $q$ limit, or when $q \dd_{\perp} \lesssim 1$
    \item size of $\varepsilon_{\bG \bG'} (\bq)$ scales much slower than 3D
    \item inverting the dielectric matrix poses no hindrance in the processing time
    \item direct inversion of $\varepsilon_{\bG \bG'} (\bq)$ gives directly $\varepsilon^{\mathrm{2D}}(q)$; Our $r_0$ agrees well with the literature
    \item exciton binding energy seems to converge; numerical coincidence?
\end{itemize}

\end{frame}