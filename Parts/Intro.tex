%!TEX root = ../main.tex %
% !TeX spellcheck = en_US

%%%%%%%  INTRODUCTION  %%%%%%%%%

\section{Introduction to Excitons in 2D materials}
%%%%%%%%%%%%%%%%%%%%%%%%%%%%%%%%%%%%%%%%%%%%%%%%%%%%%%%%%%%%%%%%%%%%

\begin{frame}{Context: Excitons and \underline{Screening}}
        \begin{columns}[T]

            \column{.5\linewidth}
            \onslide<1->{
                \textbf{...in a crystal.}

                \begin{figure}
                    \centering
                    \includegraphics[scale=1.]{Figures/hBN_permittivity.pdf}
                \end{figure}
            }

            \column{.5\linewidth}
            \onslide<2->{
                \textbf{For a typical 3D material (e.g., GaAs)}
                \begin{equation}
                    E_{b,n} = - \frac{R^*}{n^2}
                \end{equation}

                \begin{equation}
                    \label{eq:Rydberg_modified}
                    R^* = \frac{2 \mu_{eh}^2 e^4}{\hbar^2 (8 \pi \varepsilon)^2}
                \end{equation}
            }
        \end{columns}


        \begin{columns}[T]

            \column{.4\linewidth}
            \onslide<4->{
                \textbf{But for a 2D (insulating) material}
            }
            \onslide<5->{
                \begin{itemize}
                    \item Excitonic series $\neq$ Rydberg series
                    \item Dielectric "constant" $\varepsilon = \lim_{q \to 0} \lim_{\omega \to 0} \varepsilon (q, \omega) = 1$
                    \item $\varepsilon (q)$ highly dependent on $q$ (non-locality)
                \end{itemize}
            }
            \column{.6\linewidth}
            \only<6>{
                \vspace{-0.4cm}
                \begin{figure}
                    \centering
                    \includegraphics[scale=0.4]{Figures/Exciton_Series_b}
                \end{figure}
                \vspace{-0.4cm}
                Alexey Chernikov et al., Phys. Rev. Lett. 113, 076802 (2014)
            }
            \only<7>{
                \begin{itemize}
                    \item "Theoretical Methods for Excitonic Physics in 2D Materials", Maurício et al. Phys. Status Solidi B, 259: 2200097, 2022 (tutorial)
                    \item "Diele. screening in 2D insu.: Implications for excitonic and impurity states in graphane", Cudazzo et al., Phys. Rev. B \textbf{84}, 085406, 2011
                \end{itemize}
            }

        \end{columns}

\end{frame}

\begin{frame}{Motivation}
    No motivation
\end{frame}