%!TEX root = ../main.tex %
% !TeX spellcheck = en_US

%%%%%%%  SCREENING  %%%%%%%%%

\section{Part II: Screening in 2D materials}
%%%%%%%%%%%%%%%%%%%%%%%%%%%%%%%%%%%%%%%%%%%%%%%%%%%%%%%%%%%%%%%%%%%%

\subsection{Motivation}

\begin{frame}{Motivation}
    \begin{columns}[T]

        \column{.5\linewidth}
        \only<1->{
            Rytova--Keldysh model
            \begin{equation}
                \varepsilon_{\mathrm{RK}} (q) = 1 + r_0 q
            \end{equation}}
        \only<2->{
            \begin{itemize}
                \item Intuitive
                \item Analytical expression (also for $V_{\mathrm{RK}}(r)$)
                \item \textcolor{red}{Excitons are determined numerically}
                \item \textcolor{red}{Questionable for layered systems}
                \item \textcolor{red}{Screening parameter $r_0$ obtained from \textit{ab initio} methods either way}
            \end{itemize}
        }

        \column{.5\linewidth}
        \onslide<3->{
        Full numerical \textit{ab initio}
            \begin{equation}
                \varepsilon_{\vectormath{G} \vectormath{G}'} (\vectormath{q}) = \delta_{\vectormath{G} \vectormath{G}'} - v_c(\vectormath{q} + \vectormath{G}) \chi^0_{\vectormath{G} \vectormath{G}'}(\vectormath{q})
            \end{equation}

            \begin{itemize}
                \item Works for any kind of system
                \item Implementation done only once
                \item Several packages available (BerkeleyGW, Yambo, VESPA, etc...)
                \item Captures screening in its $q$ entirety
                \item \textcolor{red}{Very heavy computationally}
                \item \textcolor{red}{Technically involved/Less accessible}
                \item \textcolor{red}{Systems are always 3D}
            \end{itemize}
        }
    \end{columns}
\end{frame}

\begin{frame}{Motivation}

    \onslide<1->{(Citing J. N. S. Gomes, C. Trallero-Giner, and Mikhail I. Vasilevskiy in "Variational calculation of the lowest exciton states in phosphorene and TMDs", J.~Phys.: Condens. Matter 34 (2022) 045702)
    
    \vspace{0.5cm}
    ``There is a broad literature on the electronic and exciton properties of TMDs [...] Using the [\textit{ab initio}] Bethe--Salpeter equation on top of the DFT calculations, two-particle (including exciton) properties have been evaluated.
    However, \textbf{this approach is computationally very demanding}.
    Alternatively [...] the Wannier effective mass theory can be used to calculate the exciton energy spectrum...''}

    \vspace{0.5cm}
    
    \onslide<2->{\small The Xatu code, A.~J.~Uría-Álvarez, J.~J.~Esteve-Paredes, M. A. García-Blásquez , J. J. Palácios, Comp. Phys. Comm. 295 (2024) 109001}

    \vspace{0.2cm}

    \onslide<3->{\small WanTiBEXOS package for Python, Comp. Phys. Comm. 285 (2023) 108636}
\end{frame}

%%%%%%%%%%%%%%%%%%%%%%%%%%%%%%%%%%
\subsection{2D Dielectric function}

\begin{frame}{2D (RPA) dielectric function}

% \begin{columns}[T]

% \column{.5\linewidth}

    \onslide<1->{
     \begin{equation}
        \varepsilon^{\mathrm{RPA}}(\vectormath{r},\vectormath{r}';t,t') = \delta(\vectormath{r}-\vectormath{r}') \delta(t-t') - \int v_c(\vectormath{r}-\vectormath{r}'') \chi^0(\vectormath{r}'',\vectormath{r}';t-t'') \,\dr''
    \end{equation}
    }

    \onslide<2->{
        
        \begin{equation}
            \varepsilon_{\bG \bG'} (\bq) = \delta_{\bG \bG'} - \sqrt{v_c(\bq + \bG)} \chi^0_{\bG \bG'} (\bq) \sqrt{v_c(\bq + \bG')}\,, 
            \begin{cases}
                v_c(\bq) = \frac{e^2}{2 \varepsilon_0 |\bq|} & \text{in } \mathrm{2D} \\
                v_c(\bq) = \frac{e^2}{\varepsilon_0 |\bq|^2} & \text{in } \mathrm{3D}
            \end{cases}
        \end{equation}}
    \onslide<2->{
        For an insulator/semiconductor

        \begin{empheq}[innerbox=\fbox]{align}
            \chi^0_{\bG \bG'} (\bq) = \frac{2}{\mathcal{A}} \sum_{vc} \sum_{\bk \sigma} \frac{\bra{c,\bk} \ee^{-\ii(\bq + \bG) \vdot \br} \ket{v,\bk  + \bq} \! \bra{v,\bk  + \bq} \ee^{\ii(\bq + \bG') \vdot \br} \ket{c,\bk}}{\epsilon_{v,\bk + \bq} - \epsilon_{c,\bk} }
        \end{empheq}
        [1] Jack Deslippe et al., “BerkeleyGW: A massively parallel computer package for the calculation of the quasiparticle and optical properties of materials and nanostructures”, Computer Physics Communications 183.6 (2012)
    }

\end{frame}


\begin{frame}{Macroscopic dielectric function}

    \onslide<1->{
        \begin{equation}
                W_{\bG \bG'}(\bq) = \sqrt{v_c(\bq+\bG)} \varepsilon^{-1}_{\bG \bG'} (\bq) \sqrt{v_c(\bq+\bG')}
        \end{equation}}

    \onslide<2->{
        Ignoring local field effects ($\bG=\bG'=0$) and defining $\varepsilon_{\mathrm{M}}(\bq) = \frac{1}{\varepsilon^{-1}_{\bzero \bzero}(\bq)} \neq \varepsilon_{\bzero \bzero}(\bq)$

        \begin{equation}
            W(\bq) =  \frac{v_c(\bq)}{\varepsilon_{\mathrm{M}}(\bq)}
        \end{equation}}

    \onslide<3->{

        For a 2D semiconductor/insulator $\varepsilon_{\mathrm{M}}(\bq) = \varepsilon_{\mathrm{2D}}(\bq) \stackrel{q \to 0}{\approx} 1 + r_0 q \equiv \varepsilon_{\mathrm{RK}}(\bq)$

        \begin{equation}
            V_{\mathrm{RK}}(q) =  \frac{v_c(q)}{\varepsilon_{\mathrm{RK}}(q)} = \frac{e^2}{2 \varepsilon_0 (1 + r_0 q) q}
        \end{equation}}

\end{frame}


\subsection{Dielectric function in the Tight-Binding approximation}

\begin{frame}{Bloch states in the TB approx.}

     \begin{columns}

        \column{0.6\linewidth}
        \onslide<1->{
        Linear combination of atomic orbitals (LCAO) method

        \begin{equation}
            \psi_{n \bk} (\br) = \frac{1}{\sqrt{N}} \sum_{\bR} \ee^{\ii \bk \cdot \bR} \sum_{i \alpha} C_{i \alpha}^{n \bk} \phi_{\alpha} (\br - \bR - \bt_i)
        \end{equation}
        
        \begin{equation}
            H \psi_{n \bk} (\br) = \epsilon_{n \bk} \psi_{n \bk} (\br)
        \end{equation}}

        \onslide<2->{
        
        \begin{equation}
            H(\bk) \mathbf{C}^{n\bk} = \epsilon_{n\bk} \mathbf{C}^{n\bk}
        \end{equation}
        
        \begin{equation*}
            C^{n \bk}_{i \alpha} \to  \text{TB coefficients}
        \end{equation*}
}

        \column{0.5\linewidth}
        \onslide<3->{
            \hspace{-0.40cm}
        \begin{equation}
            H(\bk)\!\begin{bmatrix}
                       C^{n\vectormath{k}}_{1,1} \\
                       C^{n\vectormath{k}}_{1,2} \\
                       \vdots \\
                       C^{n\vectormath{k}}_{1,N^1_{o}} \\
                       C^{n\vectormath{k}}_{2,1} \\
                       \vdots \\
                       C^{n\vectormath{k}}_{2,N^2_{o}} \\
                       \vdots \\
                       C^{n\vectormath{k}}_{N_a,N^{(N_a)}_{o}-1} \\
                       C^{n\vectormath{k}}_{N_a,N^{(N_a)}_{o}}
            \end{bmatrix} \!= \!\epsilon_{n\bk}\!\begin{bmatrix}
                                                C^{n\vectormath{k}}_{1,1} \\
                                                C^{n\vectormath{k}}_{1,2} \\
                                                \vdots \\
                                                C^{n\vectormath{k}}_{1,N^1_{o}} \\
                                                C^{n\vectormath{k}}_{2,1} \\
                                                \vdots \\
                                                C^{n\vectormath{k}}_{2,N^2_{o}} \\
                                                \vdots \\
                                                C^{n\vectormath{k}}_{N_a,N^{(N_a)}_{o}-1} \\
                                                C^{n\vectormath{k}}_{N_a,N^{(N_a)}_{o}}
            \end{bmatrix}
        \end{equation}}

    \end{columns}

\end{frame}

\begin{frame}{Polarizability in the TB approximation}

    \begin{equation}
            H(\bk) \mathbf{C}^{n\bk} = \epsilon_{n\bk} \mathbf{C}^{n\bk}
    \end{equation}

    \onslide<1->{
        \begin{equation}
            \begin{split}
                \chi^0_{\bG \bG'} (\bq) &=\frac{2}{\mathcal{A}} \sum_{vc} \sum_{\bk \sigma} \frac{\bra{c,\bk} \ee^{-\ii(\bq + \bG) \vdot \br} \ket{v,\bk  + \bq} \! \bra{v,\bk  + \bq} \ee^{\ii(\bq + \bG') \vdot \br} \ket{c,\bk}}{\epsilon_{v,\bk + \bq} - \epsilon_{c,\bk} }= \\
                &=\frac{2}{\mathcal{A}} \sum_{v c} \sum_{\bk \sigma} \frac{I^{\bG}_{v \bk + \bq, c \bk} \qty(I^{\bG'}_{v \bk + \bq, c \bk})^*}{\epsilon_{v,\bk + \bq} - \epsilon_{c,\bk} }
            \end{split}
        \end{equation}}

    \onslide<2->{
        Point-like orbital approximation: $\phi^*_{\alpha} (\br - \bR - \bt_i) \phi_{\beta} (\br - \bR' - \bt_j) \approx \delta_{ij} \delta_{\alpha \beta} \delta_{\bR \bR'} \delta(\br - \bR - \bt_i) \Rightarrow$
        \begin{equation}
            \begin{split}
                \Rightarrow I^{\bG}_{n \bk, n' \bk + \bq} \equiv \bra{n,\bk} \ee^{-\ii(\bq + \bG) \vdot \br} \ket{n',\bk  +\bq} &= \int_{\mathcal{A}} \dr\, \psi^*_{n \bk } (\br)\, \ee^{-\ii(\bq + \bG) \vdot \br} \psi_{n' \bk+ \bq} (\br) = \\
                &= \sum_{i \alpha} (C_{i \alpha}^{n \bk })^* C_{i \alpha}^{n' \bk + \bq}  \ee^{- \ii (\bq + \bG)  \vdot \bt_i}
            \end{split}
        \end{equation}}

\end{frame}

\begin{frame}{Exciton computation scheme witht the Xatu code}

    \begin{enumerate}
        \item Diagonalize $H(\bk + \bq)$ and store all $\{\epsilon_{n\bk + \bq}\}$,$\{\mathbf{C}^{n\bk  + \bq}\}\, \forall \bq \in \mathrm{BZ}, \, \forall \bk \in \mathrm{BZ'}, \mathrm{BZ} \neq \mathrm{BZ}' $  \itemsep=2em
        \item Compute dielectric matrix $ \varepsilon_{\bG \bG'} (\bq) \, \forall \bq \in \mathrm{BZ}$  \itemsep=2em
        \item Invert $\varepsilon_{\bG \bG'} (\bq) \,\forall \bq \in \mathrm{BZ}$ \itemsep=2em
        \item Compute the exciton \itemsep=2em
    \end{enumerate}

\end{frame}

%%%%%%%%%%%%%%%%%%%%%%%%%%%%%%%%%%%%%%%%%

\subsection{Dielectric function: numerical results}

\begin{frame}{Example: Monolayer hBN}
    Model Hamiltonian from CRYSTAL [1]: DFT calculations in a Gaussian basis using the HSE06 functional

    \begin{figure}[H]
        \centering
        \includegraphics[width=0.5\linewidth]{Figures/hBN_HSE06_bands}
        \caption{Band structure of monolayer hBN}
    \end{figure}

    \begin{itemize}
        \item \relax [1] A~Erba et al., Journal of Chemical Theory and Computation 13.10 (2017)
    \end{itemize}

\end{frame}


\begin{frame}{Convergence of the inverse dielectric function}

    Does $\varepsilon^{-1}_{\bG \bG'} (\bq)$ converge?

    \begin{figure}[h]
        \begin{minipage}{0.5\textwidth}
            \includegraphics[width=0.8\textwidth]{Figures/inv_epsilon_convergence_hBN_panels.jpg}
        \end{minipage}%
        \begin{minipage}{0.5\textwidth}
		\caption{Convergence of the inverse dielectric matrix. Selection of inverse dielectric function matrix elements $\varepsilon^{-1}_{\bG \bG'} (\bq)$ computed at a point $\bq \neq \bzero$ in the BZ~for hBN. We display the head element in panel \pnl{a}, a diagonal body element in panel \pnl{b}, an off-diagonal body element in panel \pnl{c}, and a wing element in panel \pnl{d}. The cutoff $G_c$ increases by 3 \AA$^{-1}$ for each point, from $G_c = 3$ \AA$^{-1}$ up to $G_c = 54$~\AA$^{-1}$.}
            \label{fig:hBN_inv_epsilon_convergence}
        \end{minipage}
    \end{figure}

\end{frame}

\begin{frame}{Macroscopic dielectric function: results}

    \begin{columns}

        \column{0.3\textwidth}

        \onslide<1->{
            Defined as
            \begin{equation}
                \varepsilon^{\mathrm{2D}}(\bq) \equiv \frac{1}{\varepsilon^{-1}_{\bzero \bzero}(\bq)}
            \end{equation} }

        \onslide<3->{

        We recover:
        \begin{block}{}
            \begin{itemize}
                \item Rytova--Keldysh model
                \item $\varepsilon^{\mathrm{2D}}(\bzero) = 1$
                \item $\varepsilon^{\mathrm{2D}}(\bq) \to 1$ w/ $q \to \infty$
            \end{itemize}
        \end{block}}

        \column{0.7\textwidth}

        \onslide<2->{
            \begin{figure}[H]
                \centering
                \includegraphics[width=0.6\linewidth]{Figures/epsilon2D_abinitio_vs_q_hBN}
                \caption{The black curve was obtained by Simone~Latini, Thomas Olsen and Kristian Thygesen in Phys. Rev. B 92 (24 2015-12) through \textit{ab initio}~methods, applying a cutoff of $150$~eV for the reciprocal lattice vectors. Therein, the band structure of hBN was obtained through DFT~calculations within LDA. The dielectric function was calculated as a post-processing step. We computed the dielectric function along the symmetry line $\Gamma-K$. A vertical line indicating $q \dd_{\perp} = 1$ is included, with $\dd_{\perp} = 3.33$~\AA.}
            \end{figure}}
    \end{columns}
\end{frame}

\begin{frame}{Macroscopic dielectric function: 2D vs. \textit{Ab initio}}
    Herein \textcolor{blue}{Improve this slide w/ blocks}
    \begin{columns}
        \column{0.49\textwidth}
        \onslide<1->{
        \vspace{-6cm}

            \begin{equation}
                \varepsilon_{\bG \bG'} (\bq) = \delta_{\bG \bG'} - \frac{e^2}{2 |\bq + \bG| \varepsilon_0} \chi^0_{\bG \bG'}(\bq)
            \end{equation}
            
            with $\langle n,\bk |\ee^{-\ii (\bq + \bG) \vdot \br}|n',\bk'\rangle$ computed on the fly, then

            \begin{equation}
               \text{invert } \varepsilon_{\bG \bG'} (\bq)
            \end{equation}

            and finally, compute the 2D eff.~dielectric function as

            \begin{equation}
                \varepsilon^{\mathrm{2D}}(\bq) = \frac{1}{\varepsilon^{-1}_{\bzero \bzero} (\bq)}
            \end{equation}}
        
        \column{0.01\textwidth}
        \rule{.1mm}{1.3\textheight}

        \column{0.49\textwidth}

        \onslide<2->{
            \vspace{-4.5cm}
            \textit{Ab initio}
            \begin{equation}
                \varepsilon_{\bG \bG'} (\bq) = \delta_{\bG \bG'} - \frac{e^2}{|\bq + \bG|^2 \varepsilon_0} \chi^0_{\bG \bG'}(\bq)
            \end{equation}

            with $\langle n,\bk |\ee^{-\ii (\bq + \bG) \vdot \br}|n',\bk'\rangle$ via FFT, then do

            \begin{equation}
                \text{invert } \varepsilon_{\bG \bG'} (\bq)
            \end{equation}

            after, inverse Fourier transform on $z$ as
            \vspace{-0.3cm}
            \begin{equation}
                \varepsilon^{-1}_{\bzero \bzero}(\bq,z,z') = \frac{1}{L_{\perp}} \sum_{G_z, G_z'} \ee^{\ii G_z z} \varepsilon^{-1}_{G_z \hat{z} G_z' \hat{z}}(\bq) \ee^{-\ii G_z' z'}
            \end{equation}

            and finally $\varepsilon^{\mathrm{2D}}(\bq) = 1/\langle \varepsilon^{-1}_{\bzero \bzero}(\bq,z,z') \rangle$

            Also, $v_c(r) = 1/r\, \Theta(R_c - z)$, with $R_c \to \infty$
        }
    \end{columns}
\end{frame}

%%%%%%%%%%%%%%%%%%%%%%%%%%%%%%%%%%%%%%%%%
\subsection{Excitons}

\begin{frame}{Excitons with the Xatu code}

    Bethe--Salpeter Equation (BSE)

    \begin{empheq}[innerbox=\fbox]{equation}
        \qty(\epsilon_{c  \bk + \bQ} - \epsilon_{v \bk}) A^{\bQ}_{v c }(\bk) + \sum_{v' c',\bk'} K_{v c, v' c'}(\bk,\bk',\bQ) A^{\bQ}_{v'c'}(\bk') = E_X A^{\bQ}_{v c }(\bk)
        \label{eq:exact_Bethe_Salpeter_equation}
    \end{empheq}

    In the interaction kernel $K \equiv - (D - X)$, the direct $D$ and exchange $X$ interaction terms read

   \begin{equation}
        \begin{aligned}
            \label{eq:interaction_matrix_element_direct_reciprocal}
            D_{vc, v' c'}(\bk,\bk',\bQ) =& \int \dr \int \dr' \psi^*_{c,\bk+\bQ}(\br) \psi^*_{v',\bk'}(\br') W(\br,\br') \psi_{c',\bk'+\bQ}(\br) \psi_{v, \bk}(\br') = \\
            =& \frac{1}{\mathcal{A}} \sum_{\bG,\bG'} \qty(I^{\bG}_{c' \bk'+\bQ, c \bk+\bQ})^* W_{\bG \bG'}(\bk - \bk') I^{\bG'}_{v' \bk',v\bk}
        \end{aligned}
    \end{equation}

    and (albeit in \textit{ab initio} $X$ uses the bare potential $v_c$)
    \vspace{-0.3cm}
    \begin{equation}
        X_{vc, v' c'}(\bk,\bk',\bQ) = \int_{\mathcal{A}} \dr \int_{\mathcal{A}} \dr' \psi^*_{c,\bk+\bQ}(\br) \psi^*_{v',\bk'}(\br') W(\br,\br') \psi_{v,\bk}(\br) \psi_{c',\bk'+\bQ}(\br')
    \end{equation}

\end{frame}

\begin{frame}{Excitons in \ch{hBN}: numerical results}

    \begin{table}[h]
        \caption{This table examines the convergence of the excitonic ground state binding energy with the cutoff for the dielectric matrix, $G_c^{\varepsilon}$, and for the interaction matrix elements, $G_c^X$, always with $G_c^X < G_c^{\varepsilon}$. We have used $N_k=60^2$, $N_c=N_v=1$, and we have excluded the exchange interaction term. For the size of the regularization region, we used the radius $q_0 = 0.6 k_0$, where $k_0$ is the norm of the wavevector(s) closest to the origin. All values are in eV. $\Delta = 2.08366$~eV}
        \centering
            \begin{tabular}{cc||cccccc}
                \cline{3-8}
                &   & \multicolumn{6}{c|}{$G^{\varepsilon}_c$ (\AA$^{-1}$)}                                            \\ \cline{3-8}
                &   & 0                      & 3                      & 4                           & 5                        & 7                      & 8                       \\ \cline{8-8} \hline \hline
                \multicolumn{1}{|c|}{\multirow{4}{*}{$G^X_{c}$}} & 0 & \multicolumn{1}{c|}{X} & 0.979507               & 1.567935                    & 1.401440                 & 1.951914               & ---        \\
                \multicolumn{1}{|c|}{}                           & 3 & \multicolumn{1}{c|}{X} & 0.756373               & 1.619208                    & 1.785953                 & ---                    & ---          \\ \cline{4-4} \cline{4-4}
                \multicolumn{1}{|c|}{}                           & 4 & X                      & \multicolumn{1}{c|}{X} & 0.774599                    & 1.105537                 & ---                    & ---          \\ \cline{5-5} \cline{5-5}
                \multicolumn{1}{|c|}{}                           & 5 & X                      & X                      & \multicolumn{1}{c|}{X}      & 0.778244                 & ---                    & ---          \\ \cline{6-6} \cline{6-6}
                \multicolumn{1}{|c|}{}                           & 7 & X                      & X                      & X                           & \multicolumn{1}{c|}{X}   & 0.792309               & 1.300835          \\ \cline{7-7}
                \multicolumn{1}{|c|}{}                           & 8 & X                      & X                      & X                           & X                        & \multicolumn{1}{c|}{X} & 0.799489 \\  \cline{1-2} \cline{1-2} \cline{8-8}
            \end{tabular}
    \end{table}

    Cell with -- means that the output result does not make physical sense
\end{frame}

\begin{frame}{Monolayer \ch{MoS2} band structure}

    \begin{figure}[H]
        \centering
        \includegraphics[width=0.6\linewidth]{Figures/MoS2_bands}
        \caption{Band structure of monolayer \ch{MoS2} using CRYSTAL}
    \end{figure}

\end{frame}


\begin{frame}{Excitons in \ch{MoS2}: numerical results}

    \begin{table}[h]
        \caption{This table examines the convergence of the excitonic ground state binding energy with the cutoff for the dielectric matrix, $G_c^{\varepsilon}$, and for the interaction matrix elements, $G_c^X$, always with $G_c^X < G_c^{\varepsilon}$. We have used $N_k=60^2$, $N_c=N_v=1$, and we have excluded the exchange interaction term. For the size of the regularization region, we used the radius $q_0 = 0.6 k_0$, where $k_0$ is the norm of the wavevector(s) closest to the origin. All values are in eV. $\Delta = 2.08366$~eV}
        \centering
            \begin{tabular}{cc||cccccc}
                \cline{3-8}
                &   & \multicolumn{6}{c|}{$G^{\varepsilon}_c$ (\AA$^{-1}$)}                                            \\ \cline{3-8}
                &   & 0                      & 3                      & 4                           & 5                        & 7                      & 8                       \\ \cline{8-8} \hline \hline
                \multicolumn{1}{|c|}{\multirow{4}{*}{$G^X_{c}$}} & 0 & \multicolumn{1}{c|}{X} & 0.979507               & 1.567935                    & 1.401440                 & 1.951914               & ---        \\
                \multicolumn{1}{|c|}{}                           & 3 & \multicolumn{1}{c|}{X} & 0.756373               & 1.619208                    & 1.785953                 & ---                    & ---          \\ \cline{4-4} \cline{4-4}
                \multicolumn{1}{|c|}{}                           & 4 & X                      & \multicolumn{1}{c|}{X} & 0.774599                    & 1.105537                 & ---                    & ---          \\ \cline{5-5} \cline{5-5}
                \multicolumn{1}{|c|}{}                           & 5 & X                      & X                      & \multicolumn{1}{c|}{X}      & 0.778244                 & ---                    & ---          \\ \cline{6-6} \cline{6-6}
                \multicolumn{1}{|c|}{}                           & 7 & X                      & X                      & X                           & \multicolumn{1}{c|}{X}   & 0.792309               & 1.300835          \\ \cline{7-7}
                \multicolumn{1}{|c|}{}                           & 8 & X                      & X                      & X                           & X                        & \multicolumn{1}{c|}{X} & 0.799489 \\  \cline{1-2} \cline{1-2} \cline{8-8}
            \end{tabular}
    \end{table}

    Cell with -- means that the output result does not make physical sense
\end{frame}
