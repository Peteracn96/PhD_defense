%!TEX root = ../main.tex %
% !TeX spellcheck = en_US

%%%%%%%  SCREENING  %%%%%%%%%

\section{Part II: Screening in 2D Materials with the Xatu Code}
%%%%%%%%%%%%%%%%%%%%%%%%%%%%%%%%%%%%%%%%%%%%%%%%%%%%%%%%%%%%%%%%%%%%

%%%%%%%%%%%%%%%%%%%%%%%%%%%%%%%%%%
\subsection{2D Dielectric Function: Theory}

\begin{frame}{2D (RPA) Dielectric Function}

% \begin{columns}[T]

% \column{.5\linewidth}

    \onslide<1->{
        \begin{equation*}
            \varepsilon^{\mathrm{RPA}}(\vectormath{r},\vectormath{r}';t,t') = \delta(\vectormath{r}-\vectormath{r}') \delta(t-t') - \int v_c(\vectormath{r}-\vectormath{r}'') \chi^0(\vectormath{r}'',\vectormath{r}';t-t'') \,\dr''
        \end{equation*}
    }

    \onslide<2->{
        \vspace{-0.4cm}
        \begin{equation*}
            \Downarrow \mathcal{F} \text{ and } \omega \to 0
        \end{equation*}}
        \onslide<3->{\begin{equation*}
            \varepsilon_{\bG \bG'} (\bq) = \delta_{\bG \bG'} - \sqrt{v_c(\bq + \bG)} \chi^0_{\bG \bG'} (\bq) \sqrt{v_c(\bq + \bG')}
            \begin{cases}
                v_c(\bq) = \frac{e^2}{2 \varepsilon_0 |\bq|} & \text{in } \mathrm{2D} \\
                v_c(\bq) = \frac{e^2}{\varepsilon_0 |\bq|^2} & \text{in } \mathrm{3D}
            \end{cases}
        \end{equation*}}
    \onslide<4->{
        For an insulator/semiconductor

        \begin{empheq}[innerbox=\fbox]{align}
            \chi^0_{\bG \bG'} (\bq) = \frac{2}{\mathcal{A}} \sum_{vc} \sum_{\bk \sigma} \frac{\bra{c,\bk} \ee^{-\ii(\bq + \bG) \vdot \br} \ket{v,\bk  + \bq} \! \bra{v,\bk  + \bq} \ee^{\ii(\bq + \bG') \vdot \br} \ket{c,\bk}}{\epsilon_{v\bk + \bq} - \epsilon_{c\bk} } \notag
        \end{empheq}
        \hspace{0.7cm} Jack Deslippe et al. - \textit{Comp. Phys. Comm.} \textbf{183} 6 (2012) $\leftarrow$ BerkeleyGW
    }

\end{frame}


\begin{frame}{Macroscopic Dielectric Function}

    % \onslide<1->{
    %     Screened Coulomb potential in a crystal:
    %     $W(\br,\br') = \int \dr'' \varepsilon^{-1}(\br,\br'') v_c(\br''-\br') \xrightarrow{\mathcal{F}} W_{\bG \bG'}(\bq) = \sqrt{v_c(\bq+\bG)} \varepsilon^{-1}_{\bG \bG'} (\bq) \sqrt{v_c(\bq+\bG')}$}

    % \onslide<2->{
    %     Defining the macroscopic screened potential as:
    %     $W(\bq) \equiv W_{\bzero \bzero}(\bq) = \varepsilon^{-1}_{\bzero \bzero}(\bq) v_c(\bq) = \pause \frac{v_c(\bq)}{\varepsilon_{\mathrm{M}}(\bq)}$}
    % \onslide<3->{
    %     where we defined the macroscopic dielectric function as
    %     $\varepsilon_{\mathrm{M}}(\bq) \equiv \frac{1}{\varepsilon^{-1}_{\bzero \bzero}(\bq)}$

    %     \small Note: $\varepsilon^{-1}_{\bzero \bzero}(\bq) \text{ is the } \bzero \bzero$ element of the inverse $\neq$ inverse of the $\bzero \bzero$ element}


    \begin{itemize}\setlength{\itemsep}{10pt}
        \item<1-> Screened Coulomb potential in a crystal:
        $W(\br,\br') = \int \dr'' \varepsilon^{-1}(\br,\br'') v_c(\br''-\br') \xrightarrow{\mathcal{F}} W_{\bG \bG'}(\bq) = \sqrt{v_c(\bq+\bG)} \varepsilon^{-1}_{\bG \bG'} (\bq) \sqrt{v_c(\bq+\bG')}$
        \item<2-> Defining the macroscopic screened potential as: \\
        $W(\bq) \equiv W_{\bzero \bzero}(\bq) = \varepsilon^{-1}_{\bzero \bzero}(\bq) v_c(\bq) = \pause \frac{v_c(\bq)}{\varepsilon_{\mathrm{M}}(\bq)}$
        \item<3-> where the macroscopic dielectric function is $\varepsilon_{\mathrm{M}}(\bq) \equiv \frac{1}{\varepsilon^{-1}_{\bzero \bzero}(\bq)}$ \\
        \small Note: $\varepsilon^{-1}_{\bzero \bzero}(\bq) \text{ is the } \bzero \bzero$ element of the inverse $\neq$ inverse of the $\bzero \bzero$ element
    \end{itemize}

    \onslide<4->{
        For a 2D semiconductor/insulator, where $v_c(\bq) \sim 1/q $, we have \\

        \begin{equation*}
            \varepsilon_{\mathrm{2D}}(\bq) \equiv \varepsilon_{\mathrm{M}}(\bq) \stackrel{q \to 0}{\approx} 1 + r_0 q \equiv \varepsilon_{\mathrm{RK}}(\bq) \onslide<5->{ \Rightarrow V_{\mathrm{RK}}(q) =  \frac{v_c(q)}{\varepsilon_{\mathrm{RK}}(q)} = \frac{e^2}{2 \varepsilon_0 (1 + r_0 q) q}}
        \end{equation*}}
\end{frame}


\begin{frame}{Bloch States in the Tight-Binding Approximation}

     \begin{columns}

        \column{0.55\linewidth}
        \onslide<1->{
        Linear combination of atomic orbitals (LCAO) method

        \begin{equation*}
            \psi_{n, \bk} (\br) = \frac{1}{\sqrt{N}} \sum_{\bR} \ee^{\ii \bk \cdot \bR} \sum_{i \alpha} C_{i \alpha}^{n \bk} \phi_{\alpha} (\br - \bR - \bt_i)
        \end{equation*}
        
        \begin{equation*}
            H \psi_{n, \bk} (\br) = \epsilon_{n \bk} \psi_{n, \bk} (\br)
        \end{equation*}}

        \onslide<2->{
        
        \begin{equation*}
            H(\bk) \mathbf{C}^{n\bk} = \epsilon_{n\bk} \mathbf{C}^{n\bk}}
        \end{equation*}
        \onslide<3->{
        \begin{equation*}
           \left(\mathbf{C}^{n\bk}\right)_{i \alpha} = C^{n \bk}_{i \alpha} \to  \text{Tight-binding (TB) coefficients}
        \end{equation*}}


        \column{0.45\linewidth}
        \onslide<2->{
            \hspace{-1.5cm}
        \begin{equation*}
            H(\bk)\!\begin{bmatrix}
                       C^{n\vectormath{k}}_{1,1} \\
                       C^{n\vectormath{k}}_{1,2} \\
                       \vdots \\
                       C^{n\vectormath{k}}_{1,N^1_{o}} \\
                       C^{n\vectormath{k}}_{2,1} \\
                       \vdots \\
                       C^{n\vectormath{k}}_{2,N^2_{o}} \\
                       \vdots \\
                       C^{n\vectormath{k}}_{N_a,N^{(N_a)}_{o}-1} \\
                       C^{n\vectormath{k}}_{N_a,N^{(N_a)}_{o}}
            \end{bmatrix} \!= \!\epsilon_{n\bk}\!\begin{bmatrix}
                                                C^{n\vectormath{k}}_{1,1} \\
                                                C^{n\vectormath{k}}_{1,2} \\
                                                \vdots \\
                                                C^{n\vectormath{k}}_{1,N^1_{o}} \\
                                                C^{n\vectormath{k}}_{2,1} \\
                                                \vdots \\
                                                C^{n\vectormath{k}}_{2,N^2_{o}} \\
                                                \vdots \\
                                                C^{n\vectormath{k}}_{N_a,N^{(N_a)}_{o}-1} \\
                                                C^{n\vectormath{k}}_{N_a,N^{(N_a)}_{o}}
            \end{bmatrix}
        \end{equation*}}

    \end{columns}

\end{frame}

\begin{frame}{Polarizability in the TB Approximation}
    \onslide<2->{Recalling that $ H(\bk) \mathbf{C}^{n\bk} = \epsilon_{n\bk} \mathbf{C}^{n\bk}$ and the expression for the polarizability}
   \vspace{-0.1cm}
    \begin{equation*}
        \begin{split}
                \onslide<3->{\chi^0_{\bG \bG'} (\bq) &=\frac{2}{\mathcal{A}} \sum_{vc} \sum_{\bk \sigma} \frac{\bra{c,\bk} \ee^{-\ii(\bq + \bG) \vdot \br} \ket{v,\bk  + \bq} \! \bra{v,\bk  + \bq} \ee^{\ii(\bq + \bG') \vdot \br} \ket{c,\bk}}{\epsilon_{v\bk + \bq} - \epsilon_{c\bk} } \onslide<8>{=}\\}
                \onslide<8->{ &=\frac{2}{\mathcal{A}} \sum_{v c} \sum_{\bk \sigma} \frac{I^{\bG}_{v \bk + \bq, c \bk} \qty(I^{\bG'}_{v \bk + \bq, c \bk})^*}{\epsilon_{v\bk + \bq} - \epsilon_{c\bk} }}
        \end{split}
    \end{equation*}

    \onslide<4->{
        \textcolor{green!50!black}{Auxiliary calculation:} \\}
    \onslide<5->{
        In the point-like orbital approximation $\phi^*_{\alpha} (\br - \bR - \bt_i) \phi_{\beta} (\br - \bR' - \bt_j) \approx \delta_{ij} \delta_{\alpha \beta} \delta_{\bR \bR'} \delta(\br - \bR - \bt_i) \Rightarrow$}
        \begin{equation*}
            \begin{split}
                \onslide<6->{\bra{n,\bk} \ee^{-\ii(\bq + \bG) \vdot \br} \ket{n',\bk  +\bq} &= \int_{\mathcal{A}} \dr\, \psi^*_{n, \bk } (\br)\, \ee^{-\ii(\bq + \bG) \vdot \br} \psi_{n', \bk+ \bq} (\br) = \\}
                \onslide<7->{&= \sum_{i \alpha} (C_{i \alpha}^{n \bk })^* C_{i \alpha}^{n' \bk + \bq}  \ee^{- \ii (\bq + \bG)  \vdot \bt_i}  \equiv I^{\bG}_{n \bk, n' \bk + \bq}}
            \end{split}
        \end{equation*}

\end{frame}

\begin{frame}{Exciton Computation Scheme with the Xatu Code}
    \setbeamercovered{transparent=15}

    \begin{enumerate}\setlength{\itemsep}{20pt}
        \item<1-> Diagonalize $H(\bk + \bq)$ and store all $\{\epsilon_{n\bk + \bq}\}$,$\{\mathbf{C}^{n\bk  + \bq}\}\, \forall \bq \in \mathrm{BZ}, \, \forall \bk \in \mathrm{BZ'}, \mathrm{BZ} \neq \mathrm{BZ}' $ 
        \item<2-> Compute dielectric matrix $ \varepsilon_{\bG \bG'} (\bq) \, \forall \bq \in \mathrm{BZ}$  
        \item<3-> Invert $\varepsilon_{\bG \bG'} (\bq) \,\forall \bq \in \mathrm{BZ}$ 
        \item<4-> Compute the exciton (following slides)
    \end{enumerate}
    \vspace{0.3cm}

    A.~J.~Uría-Alvárez et al. - \textit{Comp. Phys. Comm.} \textbf{295} 109001 (2024) $\leftarrow$ 1st version of Xatu
\end{frame}

%%%%%%%%%%%%%%%%%%%%%%%%%%%%%%%%%%%%%%%%%

\subsection{2D Dielectric Function and Excitons: Results}

\begin{frame}{Example: Monolayer hBN}
    Model Hamiltonian from CRYSTAL [1]: DFT calculations in a Gaussian basis using the HSE06 functional

    \begin{figure}[H]
        \centering
        \includegraphics[width=0.5\linewidth]{Figures/hBN_HSE06_bands}
        \caption{Band structure of monolayer hBN}
    \end{figure}

    \begin{itemize}
        \item \relax [1] A~Erba et al. - \textit{J. Chem. Theory Comput.} \textbf{13} 10 (2017)
    \end{itemize}

\end{frame}


\begin{frame}{Macroscopic Dielectric Function: Results}
    \pdfpcnote{when exciton, say that}
    \pdfpcnote{question left open in the thesis}
    \pdfpcnote{which we address in this talk}

    \begin{columns}
        \column{.45\linewidth}
        \onslide<1->{Does $\varepsilon^{-1}_{\bG \bG'} (\bq)$ converge?} \onslide<3->{\textcolor{red}{No!}}

        \onslide<2->{
        \begin{figure}[h]
            \begin{minipage}{\textwidth}
                \includegraphics[width=1.0\textwidth]{Figures/inv_epsilon_convergence_hBN_panels.jpg}
            \end{minipage}%
            \begin{minipage}{0.5\textwidth}		
            \end{minipage}
        \end{figure}}
        
       
        \column{.55\linewidth}

        \onslide<4->{
            But still,
            \begin{itemize}
                \item<5-> How close can we get to the \textit{ab initio} result?
                \item<7-> Does the exciton binding energy converge?
            \end{itemize}
            \
        }
        \onslide<6->{   
            \vspace{-0.8cm}
            \begin{figure}[H]
                \centering
                \includegraphics[width=\linewidth]{Figures/epsilon2D_abinitio_vs_q_hBN}                
            \end{figure}
            \vspace{-0.3cm}
            \small S.~Latini, T.~Olsen, K.~S. Thygesen - \textit{Phys. Rev. B} \textbf{92} 245123\\ \hspace{7.3cm} (2015)}
    \end{columns}    
\end{frame}

% \begin{frame}{Macroscopic dielectric function: results}

%     \begin{columns}

%         \column{0.3\textwidth}

%         \onslide<1->{
%             Defined as
%             \begin{equation*}
%                 \varepsilon^{\mathrm{2D}}(\bq) \equiv \frac{1}{\varepsilon^{-1}_{\bzero \bzero}(\bq)}
%             \end{equation*} }

%         \onslide<3->{

%         \textbf{We recover:}
%         \begin{block}{}
%             \begin{itemize}
%                 \item Rytova--Keldysh model
%                 \item $\varepsilon^{\mathrm{2D}}(\bzero) = 1$
%                 \item $\varepsilon^{\mathrm{2D}}(\bq) \to 1$ w/ $q \to \infty$
%             \end{itemize}
%         \end{block}}

%         \column{0.7\textwidth}

%         \onslide<2->{
%             \begin{figure}[H]
%                 \centering
%                 \includegraphics[width=0.6\linewidth]{Figures/epsilon2D_abinitio_vs_q_hBN}                
%             \end{figure}
%             S.~Latini, T.~Olsen, K.~S. Thygesen - \textit{Phys. Rev. B} \textbf{92} 245123 (2015)}
%     \end{columns}
% \end{frame}

\begin{frame}{Macroscopic Dielectric Function: 2D vs. \textit{Ab initio}}
    \setbeamercovered{transparent=15}
    \begin{columns}
        \column{0.49\textwidth}
        \vspace{-6.8cm}
        \onslide<1->{
            \begin{block}{2D approach}
            %\vspace{-6cm}
                \begin{enumerate}\setlength{\itemsep}{10pt}
                    \item<2-> $\varepsilon_{\bG \bG'} (\bq) = \delta_{\bG \bG'} - \frac{e^2}{2 |\bq + \bG| \varepsilon_0} \chi^0_{\bG \bG'}(\bq)$ \\
                          w/ $\langle n,\bk |\ee^{-\ii (\bq + \bG) \vdot \br}|n',\bk'\rangle$ computed on the fly
                    \item<3-> invert $ \varepsilon_{\bG \bG'} (\bq)$
                    \item<4-> pick the head element $\varepsilon^{\mathrm{2D}}(\bq) = \frac{1}{\varepsilon^{-1}_{\bzero \bzero} (\bq)}$
                \end{enumerate}
        \end{block}}
        
        \column{0.01\textwidth}
        \rule{.1mm}{1.3\textheight}

        \column{0.49\textwidth}

        \onslide<5->{
            \vspace{-5.cm}
            \begin{block}{\textit{Ab initio}}
            %\vspace{-6.0cm}
                \begin{enumerate}\setlength{\itemsep}{10pt}
                    \item<6-> $\varepsilon_{\bG \bG'} (\bq) = \delta_{\bG \bG'} - \frac{e^2}{|\bq + \bG|^2 \varepsilon_0} \chi^0_{\bG \bG'}(\bq)$
                    w/ $\langle n,\bk |\ee^{-\ii (\bq + \bG) \vdot \br}|n',\bk'\rangle$ via FFT
                    \item<7-> invert $ \varepsilon_{\bG \bG'} (\bq)$
                    \item<8-> $\varepsilon^{-1}_{\bzero \bzero}(\bq,z,z') = \frac{1}{L_{\perp}} \sum_{G_z, G_z'} \ee^{\ii G_z z} \varepsilon^{-1}_{G_z \hat{z} G_z' \hat{z}}(\bq) \ee^{-\ii G_z' z'}$
                    \item<9-> $\varepsilon^{\mathrm{2D}}(\bq) = 1/\langle \varepsilon^{-1}_{\bzero \bzero}(\bq,z,z') \rangle_{\text{off-plane}}$
                \end{enumerate}
                \vspace{0.3cm}
                \onslide<10->{Also, the Coulomb potential is truncated as\\
                 $v_c(\br) = \frac{\Theta(R_c - z)}{r}$, with $R_c \to \infty$}
                \end{block}
        }
    \end{columns}
\end{frame}

%%%%%%%%%%%%%%%%%%%%%%%%%%%%%%%%%%%%%%%%%

\begin{frame}{Excitons with the Xatu Code}

    \onslide<1->{Exact diagonalization of the Bethe--Salpeter Equation (BSE)

    \begin{empheq}[innerbox=\fbox]{equation*}
        \qty(\epsilon_{c  \bk + \bQ} - \epsilon_{v \bk}) A^{\bQ}_{v c }(\bk) + \sum_{v' c',\bk'} K_{v c, v' c'}(\bk,\bk',\bQ) A^{\bQ}_{v'c'}(\bk') = E_X A^{\bQ}_{v c }(\bk)
        \label{eq:exact_Bethe_Salpeter_equation}
    \end{empheq}}

    \onslide<2->{In the interaction kernel $K \equiv - (D - X)$, the direct $D$ and exchange $X$ interaction terms read}

   \begin{equation*}
        \begin{aligned}
            \label{eq:interaction_matrix_element_direct_reciprocal}
            \onslide<3->{D_{vc, v' c'}(\bk,\bk',\bQ) =& \int \dr \int \dr' \psi^*_{c,\bk+\bQ}(\br) \psi^*_{v',\bk'}(\br') W(\br,\br') \psi_{c',\bk'+\bQ}(\br) \psi_{v, \bk}(\br') = \\
            =& \frac{1}{\mathcal{A}} \sum_{\bG,\bG'} \qty(I^{\bG}_{c' \bk'+\bQ, c \bk+\bQ})^* W_{\bG \bG'}(\bk - \bk') I^{\bG'}_{v' \bk',v\bk}}
        \end{aligned}
    \end{equation*}

    \onslide<3->{and}

    \vspace{-0.3cm}
    \begin{equation*}
        \onslide<4->{X_{vc, v' c'}(\bk,\bk',\bQ) = \int_{\mathcal{A}} \dr \int_{\mathcal{A}} \dr' \psi^*_{c,\bk+\bQ}(\br) \psi^*_{v',\bk'}(\br') W(\br,\br') \psi_{v,\bk}(\br) \psi_{c',\bk'+\bQ}(\br') \to 0}
    \end{equation*}

\end{frame}

\begin{frame}{Excitons in \ch{hBN}: Numerical Results}
    \pdfpcnote{When red, go back to slide 25}
    \pdfpcnote{then come back to 27}
    \only<1>{
    \begin{table}[h]
        \caption{$N_k=60^2$, $N_c=N_v=1$, All values are in eV.}
        \centering
        \begin{tabular}{cc||cccccc}
            \cline{3-8}
            &   & \multicolumn{6}{c|}{$G^{\varepsilon}_c$}                                            \\ \cline{3-8}
            &                                                & 0                          & 3                & 5.1              & 6              & 8                      & 9                       \\ \cline{8-8} \hline \hline
            \multicolumn{1}{|c|}{\multirow{4}{*}{$G^X_{c}$}} & 0 & \multicolumn{1}{c|}{\phantom{X}} & 4.14935                & 5.5566                      & 5.75416                  & 6.011076               & $E_b >$ gap        \\
            \multicolumn{1}{|c|}{}                           & 3 & \multicolumn{1}{c|}{\phantom{X}} & 2.503138               & 4.380231                    & 3.803146                 & 3.811565               & 4.326392          \\ \cline{4-4} \cline{4-4}
            \multicolumn{1}{|c|}{}                           & 5.1 & \phantom{X}                    & \multicolumn{1}{c|}{\phantom{X}} & 2.797365                    & 2.99418                  & 3.888089               & 4.630105          \\ \cline{5-5} \cline{5-5}
            \multicolumn{1}{|c|}{}                           & 6 & \phantom{X}                      & \phantom{X}                      & \multicolumn{1}{c|}{\phantom{X}}      & 2.83405                  & 3.492424               & 4.330678          \\ \cline{6-6} \cline{6-6}
            \multicolumn{1}{|c|}{}                           & 8 & \phantom{X}                      & \phantom{X}                      & \phantom{X}                           & \multicolumn{1}{c|}{\phantom{X}}   & 2.900768               & 5.811998          \\ \cline{7-7}
            \multicolumn{1}{|c|}{}                           & 9 & \phantom{X}                      & \phantom{X}                      & \phantom{X}                           & \phantom{X}                        & \multicolumn{1}{c|}{\phantom{X}} & 3.018396 \\  \cline{1-2} \cline{1-2} \cline{8-8}
        \end{tabular}
    \end{table}
    
    \begin{itemize}
        \item \phantom{For $G^X_c = 0$, $E_b$ increases with increasing $G^{\varepsilon}_c$ (decreased screening)}
        \item \phantom{For a given $G^{\varepsilon}_c$, $E_b$ converges w/ increasing $G^X_c$ (somewhat)}
        \item \phantom{Quite remarkably, $E_b$ converges!}
    \end{itemize}}

    \only<2>{
    \begin{table}[h]
        \caption{$N_k=60^2$, $N_c=N_v=1$, All values are in eV.}
        \centering
        \begin{tabular}{cc||cccccc}
            \cline{3-8}
            &   & \multicolumn{6}{c|}{$G^{\varepsilon}_c$}                                            \\ \cline{3-8}
            &                                                & 0                          & 3                & 5.1              & 6              & 8                      & 9                       \\ \cline{8-8} \hline \hline
            \multicolumn{1}{|c|}{\multirow{4}{*}{$G^X_{c}$}} & \textcolor{red}{0} & \multicolumn{1}{c|}{\phantom{X}} & \textcolor{red}{4.14935}                & \textcolor{red}{5.5566}                      & \textcolor{red}{5.75416}                  & \textcolor{red}{6.011076}               & $E_b >$ gap        \\
            \multicolumn{1}{|c|}{}                           & 3 & \multicolumn{1}{c|}{\phantom{X}} & 2.503138               & 4.380231                    & 3.803146                 & 3.811565               & 4.326392          \\ \cline{4-4} \cline{4-4}
            \multicolumn{1}{|c|}{}                           & 5.1 & \phantom{X}                    & \multicolumn{1}{c|}{\phantom{X}} & 2.797365                    & 2.99418                  & 3.888089               & 4.630105          \\ \cline{5-5} \cline{5-5}
            \multicolumn{1}{|c|}{}                           & 6 & \phantom{X}                      & \phantom{X}                      & \multicolumn{1}{c|}{\phantom{X}}      & 2.83405                  & 3.492424               & 4.330678          \\ \cline{6-6} \cline{6-6}
            \multicolumn{1}{|c|}{}                           & 8 & \phantom{X}                      & \phantom{X}                      & \phantom{X}                           & \multicolumn{1}{c|}{\phantom{X}}   & 2.900768               & 5.811998          \\ \cline{7-7}
            \multicolumn{1}{|c|}{}                           & 9 & \phantom{X}                      & \phantom{X}                      & \phantom{X}                           & \phantom{X}                        & \multicolumn{1}{c|}{\phantom{X}} & 3.018396 \\  \cline{1-2} \cline{1-2} \cline{8-8}
        \end{tabular}
    \end{table}}

    \only<3>{
    \begin{table}[h]
        \caption{$N_k=60^2$, $N_c=N_v=1$, All values are in eV.}
        \centering
        \begin{tabular}{cc||cccccc}
            \cline{3-8}
            &   & \multicolumn{6}{c|}{$G^{\varepsilon}_c$}          \\ \cline{3-8}
            &                                                & 0                          & 3                      & 5.1              & \textcolor{green!50!black}{6}              & 8                      & 9                       \\ \cline{8-8} \hline \hline
            \multicolumn{1}{|c|}{\multirow{4}{*}{$G^X_{c}$}} & 0 & \multicolumn{1}{c|}{\phantom{X}} & 4.14935                & 5.5566                     & \textcolor{green!50!black}{5.75416}                  & 6.011076               & $E_b >$ gap        \\
            \multicolumn{1}{|c|}{}                           & 3 & \multicolumn{1}{c|}{\phantom{X}} & 2.503138               & 4.380231                    & \textcolor{green!50!black}{3.803146}                 & 3.811565               & 4.326392          \\ \cline{4-4} \cline{4-4}
            \multicolumn{1}{|c|}{}                           & 5.1 & \phantom{X}                    & \multicolumn{1}{c|}{\phantom{X}} & 2.797365                    & \textcolor{green!50!black}{2.99418}                 & 3.888089               & 4.630105          \\ \cline{5-5} \cline{5-5}
            \multicolumn{1}{|c|}{}                           & 6 & \phantom{X}                      & \phantom{X}                      & \multicolumn{1}{c|}{\phantom{X}}      & \textcolor{green!50!black}{2.83405}                  & 3.492424               & 4.330678          \\ \cline{6-6} \cline{6-6}
            \multicolumn{1}{|c|}{}                           & 8 & \phantom{X}                      & \phantom{X}                      & \phantom{X}                           & \multicolumn{1}{c|}{\phantom{X}}   & 2.900768               & 5.811998          \\ \cline{7-7}
            \multicolumn{1}{|c|}{}                           & 9 & \phantom{X}                      & \phantom{X}                      & \phantom{X}                           & \phantom{X}                        & \multicolumn{1}{c|}{\phantom{X}} & 3.018396 \\  \cline{1-2} \cline{1-2} \cline{8-8}
        \end{tabular}
    \end{table}}

    \only<4>{
    \begin{table}[h]
        \caption{$N_k=60^2$, $N_c=N_v=1$, All values are in eV.}
        \centering
        \begin{tabular}{cc||cccccc}
            \cline{3-8}
            &   & \multicolumn{6}{c|}{$G^{\varepsilon}_c$}                                            \\ \cline{3-8}
            &                                                & 0                          & 3                & 5.1              & 6              & 8                      & 9                       \\ \cline{8-8} \hline \hline
            \multicolumn{1}{|c|}{\multirow{4}{*}{$G^X_{c}$}} & 0 & \multicolumn{1}{c|}{\phantom{X}} & 4.14935                & 5.5566                     & 5.75416                  & 6.011076               & $E_b >$ gap        \\
            \multicolumn{1}{|c|}{}                           & 3 & \multicolumn{1}{c|}{\phantom{X}} & \textcolor{blue!50!white}{2.503138}                  & 4.380231                    & 3.803146                 & 3.811565               & 4.326392          \\ \cline{4-4} \cline{4-4}
            \multicolumn{1}{|c|}{}                           & 5.1 & \phantom{X}                    & \multicolumn{1}{c|}{\phantom{X}} & \textcolor{blue!50!white}{2.797365}                    & 2.99418                 & 3.888089               & 4.630105          \\ \cline{5-5} \cline{5-5}
            \multicolumn{1}{|c|}{}                           & 6 & \phantom{X}                      & \phantom{X}                      & \multicolumn{1}{c|}{\phantom{X}}      & \textcolor{blue!50!white}{2.83405}                  & 3.492424               & 4.330678          \\ \cline{6-6} \cline{6-6}
            \multicolumn{1}{|c|}{}                           & 8 & \phantom{X}                      & \phantom{X}                      & \phantom{X}                           & \multicolumn{1}{c|}{\phantom{X}}   & \textcolor{blue!50!white}{2.900768}               & 5.811998          \\ \cline{7-7}
            \multicolumn{1}{|c|}{}                           & 9 & \phantom{X}                      & \phantom{X}                      & \phantom{X}                           & \phantom{X}                        & \multicolumn{1}{c|}{\phantom{X}} & \textcolor{blue!50!white}{3.018396} \\  \cline{1-2} \cline{1-2} \cline{8-8}
        \end{tabular}
    \end{table}}
    
    \only<2>{
    \begin{itemize}
        \item \textcolor{red}{For $G^X_c = 0$, $E_b$ increases with increasing $G^{\varepsilon}_c$ (decreased screening)}
        \item \phantom{For a given $G^{\varepsilon}_c$, $E_b$ converges w/ increasing $G^X_c$ (somewhat)}
        \item \phantom{something}
    \end{itemize}}

    \only<3>{
    \begin{itemize}
        \item For $G^X_c = 0$, $E_b$ increases with increasing $G^{\varepsilon}_c$ (decreased screening)
        \item \textcolor{green!50!black}{For a given $G^{\varepsilon}_c$, $E_b$ converges w/ increasing $G^X_c$ (somewhat)} 
        \item \phantom{Something}
    \end{itemize}}
    
    \only<4>{
    \begin{itemize}
        \item For $G^X_c = 0$, $E_b$ increases with increasing $G^{\varepsilon}_c$ (decreased screening)
        \item For a given $G^{\varepsilon}_c$, $E_b$ converges w/ increasing $G^X_c$ (somewhat)
        \item \textcolor{blue!50!white}{Quite remarkably, $E_b$ converges!}
    \end{itemize}}

    $E_b >$ gap -> the exciton comes with negative excitation energy
\end{frame}

\iffalse
\begin{frame}{Monolayer \ch{MoS2} band structure}

    \begin{figure}[H]
        \centering
        \includegraphics[width=0.6\linewidth]{Figures/MoS2_bands}
        \caption{Band structure of monolayer \ch{MoS2} using CRYSTAL}
    \end{figure}

\end{frame}


\begin{frame}{Excitons in \ch{MoS2}: numerical results}

    \begin{table}[h]
        \caption{This table examines the convergence of the excitonic ground state binding energy with the cutoff for the dielectric matrix, $G_c^{\varepsilon}$, and for the interaction matrix elements, $G_c^X$, always with $G_c^X < G_c^{\varepsilon}$. We have used $N_k=60^2$, $N_c=N_v=1$, and we have excluded the exchange interaction term. For the size of the regularization region, we used the radius $q_0 = 0.6 k_0$, where $k_0$ is the norm of the wavevector(s) closest to the origin. All values are in eV. $\Delta = 2.08366$~eV}
        \centering
            \begin{tabular}{cc||cccccc}
                \cline{3-8}
                &   & \multicolumn{6}{c|}{$G^{\varepsilon}_c$ (\AA$^{-1}$)}                                            \\ \cline{3-8}
                &   & 0                      & 3                      & 4                           & 5                        & 7                      & 8                       \\ \cline{8-8} \hline \hline
                \multicolumn{1}{|c|}{\multirow{4}{*}{$G^X_{c}$}} & 0 & \multicolumn{1}{c|}{X} & 0.979507               & 1.567935                    & 1.401440                 & 1.951914               & ---        \\
                \multicolumn{1}{|c|}{}                           & 3 & \multicolumn{1}{c|}{X} & 0.756373               & 1.619208                    & 1.785953                 & ---                    & ---          \\ \cline{4-4} \cline{4-4}
                \multicolumn{1}{|c|}{}                           & 4 & X                      & \multicolumn{1}{c|}{X} & 0.774599                    & 1.105537                 & ---                    & ---          \\ \cline{5-5} \cline{5-5}
                \multicolumn{1}{|c|}{}                           & 5 & X                      & X                      & \multicolumn{1}{c|}{X}      & 0.778244                 & ---                    & ---          \\ \cline{6-6} \cline{6-6}
                \multicolumn{1}{|c|}{}                           & 7 & X                      & X                      & X                           & \multicolumn{1}{c|}{X}   & 0.792309               & 1.300835          \\ \cline{7-7}
                \multicolumn{1}{|c|}{}                           & 8 & X                      & X                      & X                           & X                        & \multicolumn{1}{c|}{X} & 0.799489 \\  \cline{1-2} \cline{1-2} \cline{8-8}
            \end{tabular}
    \end{table}

    Cell with -- means that the output result does not make physical sense

    At least apparent numerical convergence
\end{frame}
\fi

\subsection{Quasi-2D Approach for Screening}

\begin{frame}{Quasi-2D Approach for Screening}
    
    \onslide<1->{Introducing the bare Coulomb potential $v_c$ in the mixed $(\bq,z)$-representation}

    \begin{equation*}
        \onslide<2->{v_c(\br-\br') = v_c(\br_{||} - \br_{||}', z-z')} \onslide<3->{\xrightarrow{\mathcal{F}_{||}} v_{c}(\bq,z-z') =  \frac{e}{2 \varepsilon_0 q} \, \ee^{- q|z-z'|}}
    \end{equation*}

    \onslide<4->{Within a quasi-2D (Q2D) framework, the diele. function 

    \begin{equation*}
        \varepsilon(\vectormath{r},\vectormath{r}') = \delta(\vectormath{r}-\vectormath{r}')  - \int v_c(\vectormath{r}-\vectormath{r}'') \chi^0(\vectormath{r}'',\vectormath{r}') \,\dr''
    \end{equation*}}
    \vspace{-0.3cm}
    \begin{equation*}
            \onslide<5->{\Downarrow \mathcal{F}_{||}}
        \end{equation*}
        \vspace{-0.5cm}
    \begin{equation*}
        \onslide<6->{\varepsilon_{\bG \bG'}(\bq,z,z') = \delta_{\bG \bG'} \delta( z-z')- \frac{e}{2\varepsilon_0|\bq +\bG|} \int_{-\infty}^{\infty} \dd z'' \ee^{-|\bq +\bG||z-z''|} \chi^{0}_{\bG \bG'}(\bq,z'',z')}
    \end{equation*}

    \begin{equation*}
        \onslide<7->{\text{where~~~} \chi^{0}_{\bG \bG'}(\bq,z,z') = \int_{\mathcal{A}} \dd \br_{||} \int_{\mathcal{A}} \dd \br_{||}' \, \ee^{-\ii (\bq + \bG) \vdot \br} \chi^{0}(\br,\br') \ee^{\ii (\bq' + \bG') \vdot \br'}}
    \end{equation*}
\end{frame}

\begin{frame}{Q2D Dielectric Function}
    Defining an effective 2D macroscopic dielectric function by averaging over $\dd_{\perp}$
    
    \onslide<2->{
    \begin{equation*}
        \bar{\varepsilon}_{\bG \bG'}(\bq) \equiv \frac{1}{\dd_{\perp}} \int_{-\dd_{\perp}/2}^{\dd_{\perp}/2} \int_{-\dd_{\perp}/2}^{\dd_{\perp}/2} \varepsilon_{\bG \bG'}(\bq,z,z')\,\dd z\dd z'
    \end{equation*}}
    \onslide<3->{It is possible to show that}
    \onslide<4->{
    \begin{empheq}[innerbox=\fbox]{align}
        \label{eq:Q2D_dielectric_function_zero_thickness_limit}
        \varepsilon_{\bG \bG'}(\bq) = \lim_{\dd_{\perp}\! \to 0} \frac{1}{\dd_{\perp}} \int_{-\dd_{\perp}/2}^{\dd_{\perp}/2} \int_{-\dd_{\perp}/2}^{\dd_{\perp}/2} \varepsilon_{\bG \bG'}(\bq,z,z')\,\dd z\dd z' \notag
    \end{empheq}}
    
    \onslide<5->{Technically, $\varepsilon(\br,\br') \approx \varepsilon^{\mathrm{2D}}(\br_{||},\br_{||}') \delta(z-z')$ \Leftrightarrow \, $\dd_{\perp} \approx 0$}
    
    \vspace{0.5cm}

    \onslide<6->{But does $\bar{\varepsilon}_{\bG \bG'}(\bq)$ improve our dielectric function?}
\end{frame}

\begin{frame}{Q2D Dielectric Function: Results}
    \begin{columns}
        \column{0.6\textwidth}
        \vspace{-0.3cm}
        \hspace{-0.5cm}
        \onslide<2->{
        \begin{figure}[h]
            \centering
            \includegraphics[width=0.85\textwidth]{Figures/epsilon_M_vs_q_hBN_&_MoS2.pdf}        
        \end{figure}}

         \onslide<3->{\small [1] K.~Andersen et al. - \textit{Nano Lett.} \textbf{15} 7 (2015)}
               
        \column{0.43\textwidth}

        \onslide<3->{\small QEH $\to$ Python package for $\varepsilon(q)$ of van der Waals heterostructures [1]}

        \begin{itemize}
            \item<4-> Better agreement overall
            \item<5-> \textit{Ab initio}, 2D and Q2D approaches all agree in the small-$q$ limit!
            \item<6-> $r_0$ can be very well estimated with vanishing computational cost!
            \item<7-> Very good agreement with the literature! \\For \ch{MoS2} $r_0 \approx 32$ \AA; for \ch{GeS} $r_0 \approx 20$ \AA
        \end{itemize}
        
    \end{columns}
\end{frame}
